\documentclass[10pt,a4paper]{article}
\usepackage[latin1]{inputenc}
\usepackage{amsmath}
\usepackage{amsfonts}
\usepackage{amssymb}
\usepackage[]{theorem}
\usepackage[nottoc]{tocbibind}
\usepackage{hyperref}


\setlength{\topmargin}{-.5in}
\setlength{\textheight}{9in}
\setlength{\oddsidemargin}{.125in}
\setlength{\textwidth}{6.25in}

\theoremstyle{plain} \newtheorem{Lem}{Lemma}


\newcommand {\subsubsubsection} [1] {\vskip 0.5em {\bf \em #1}}

%\newcommand {\comm} [1] {}
\newcommand {\comm} [1]
{
\par
\setlength{\leftskip}{-0.8in}
\begin{tabular} {p{0.1cm} | p{\textwidth}}
  {\bf ?} & {#1}
\end{tabular}
\par
\setlength{\leftskip}{0in}
}


\newcommand {\cond} {, & \textrm{if }}
\newcommand {\otherwise} {, & \textrm{else}}

\newcommand {\Forall} {\forall \ }
\newcommand {\eqdef} {\triangleq}
\newcommand {\equivalent} {\Leftrightarrow}
\renewcommand {\implies} {\Rightarrow}
\renewcommand {\impliedby} {\Leftarrow}
\renewcommand {\And}  {\ \& \ }
\newcommand {\Or}  {\ \vee \ }
\newcommand {\QED} {$\square$ \ \par }
\newcommand {\proofword}  {{\par \sf Proof:\ }}
\newcommand {\proof} [1] {\proofword #1 \QED \par}
\newcommand {\definition} [2] {\bigskip {\bf Definition} {\rm \sc #1} \par #2 \QED \bigskip}
\newcommand {\example} [2] {\bigskip {\bf Example.} {#1} \par #2 \QED}
\newcommand {\theorem} [2] {\bigskip {\bf Theorem} {\rm \sc #1} \par #2 \QED \bigskip}
\newcommand {\ve} {\mathbf}
\newcommand {\vecc} {\boldsymbol}
\newcommand {\Questions} {\vspace{5mm} {\bf Unresolved Questions:}}
\newcommand {\assert} {\square \ }

\renewcommand {\P} {\mathrm P}
\newcommand {\E} {\mathrm E \ }
\newcommand {\var} {\mathrm {var} \ }
\newcommand {\cov} {\mathrm {cov} }
\newcommand {\cor} {\mathrm {cor} }
\newcommand {\N} {\mathbb N}
\newcommand {\Z} {\mathrm Z}
\newcommand {\R} {\mathbb R}
\newcommand {\C} {\mathbb C}
\newcommand {\I} {\mathbb I}
\newcommand {\supp} {\mathrm {Supp}}
\newcommand {\mat} {\mathrm}
\newcommand {\trc} {\mathrm {tr} \ }
\newcommand {\VC} {\mathrm {VC} }
\newcommand {\const} {\mathrm {const} }
\newcommand {\sgn} {\mathrm {sgn} }
\newcommand {\dom} {\mathrm {dom} }
\newcommand {\range} {\mathrm {range} }
\newcommand {\Arg} {\mathrm {Arg} \ }

\newcommand {\LCA} {\mathrm {LCA} }  % least common ancestor (in a tree)

\newcommand {\nul} {\mathrm {null}}
\newcommand {\Dt} {\Delta t}
\newcommand {\PC} {\mathrm {PC} }
\newcommand {\MDS} {\mathrm {MDS} }
\newcommand {\CC} {\mathrm {CC} }

\newcommand {\re} {\mathrm {re}}
\newcommand {\cl} {\mathrm {cl}}
\newcommand {\intr} {\mathrm {int}}

\newcommand {\<} {\langle}
\renewcommand {\>} {\rangle}

\newcommand {\dotve} [1] {\dot {\ve #1}}

\newcommand {\la} [1] {\lambda (#1) \ }

\newcommand {\lhs} {\mathrm {lhs}}
\newcommand {\rhs} {\mathrm {rhs}}

\providecommand{\keywords}[1]
{
  \small
  \textbf{\textit{Keywords---}} #1
}

\makeatletter
\DeclareRobustCommand{\cev}[1]{%
  {\mathpalette\do@cev{#1}}%
}
\newcommand{\do@cev}[2]{%
  \vbox{\offinterlineskip
    \sbox\z@{$\m@th#1 x$}%
    \ialign{##\cr
      \hidewidth\reflectbox{$\m@th#1\vec{}\mkern4mu$}\hidewidth\cr
      \noalign{\kern-\ht\z@}
      $\m@th#1#2$\cr
    }%
  }%
}
\makeatother




\title{Data Analysis}
\author{Vyacheslav Brover}


\begin{document}
\maketitle

\tableofcontents


\section{General notation}
Let the "prime" sign mean transpose.

Let capital letters denote random variables.

Let~$\R$ be the set of real numbers.
$$ \R_+ \eqdef \{x \in \R: x \ge 0 \}. $$
$$ \R_{++} \eqdef \{x \in \R: x > 0 \}. $$

Let~$\I$ be the set of imaginary numbers.

Let~$\C$ be the set of complex numbers.

$$ [n] \eqdef \{i \in \N \mid 1 \le i \le n\}. $$



\section{Distance and similarity}

\subsection{Distance}
A {\em dissimilarity} on a set~$X$ is a function $d:X,X \mapsto \C$ such that
$$ d^2(x,y) \in \R, $$
$$ d(x,y) = d(y,x) $$
and
$$ d(x,x) = 0. $$

The elements~$x$ and~$y$ are {\em indiscernible} iff $d(x,y) \le 0$.

A dissimilarity is a {\em semidistance} iff
\begin{equation} \tag{semidistance non-negativity}
  d^2(x,y) \ge 0.
\end{equation}

A semidistance is a {\em distance} iff
$$ d(x,y) \ge 0 $$
and
\begin{equation} \tag{triangle inequality}
  d(x,y) \le d(x,z) + d(z,y).
\end{equation}

If~$d$ is a distance then the indiscernible elements make up an equivalence relation on~$X$.

If $d(x,y)$ is a distance then $d^p(x,y)$, where $0 < p \le 1$ is a distance.

If~$d$ is a semidistance and
$$ \delta \eqdef \max [d(x,y) - (d(x,z) + d(z,y))], $$
then
$$ d(x,y) - (d(x,z) + d(z,y)) \le \delta $$
and
\begin{equation*}
  d_\delta(x,y) \eqdef
    \begin{cases}
      0, &\text{ if } x = y\\
      d(x,y) + \delta,   &\text{ else}
    \end{cases}
\end{equation*}
is a distance.

The {\em between-set}
$$ Inter(x,y) \eqdef \{z \in X: d(x,z) = d(z,y)\}. $$
The between-set $Inter(x,y)$ is {\em unidimensional} iff
$$ \forall r \in \R_+ : |\{z \in Inter(x,y): d(x,z) = r\}| \le 1. $$

For $a,b,c,d \in X$ let a {\em split} be a partitioning of $\{a,b,c,d\}$.
There are 3~possible partitionings.
For a partitioning $\{\{a,b\},\{c,d\}\}$
$$ d(ab|cd) \eqdef d(a,b) + d(cd). $$

A distance is {\em additive}, see \cite{Buneman}, iff
\begin{equation} \tag{additivity}
\forall a,b,c,d \in X : d(ab|cd) \le \max \{d(ac|bd), d(ad|bc)\}.
\end{equation}

The triangle inequality follows form the additivity property.


\subsection{Similarity}

A {\em comparison} on a set~$X$ is a function $s:X,X \mapsto \R$ such that

$$ s(x,y) = s(y,x). $$

$$ s(x) \eqdef s(x,x). $$

The {\em center} of a comparison~$s$
$$ C(s) \eqdef \{x \in X \mid s(x) = 0\}. $$

A comparison is a {\em semisimilarity} iff
$$ s(x) \ge 0. $$

A semisimilarity is a {\em similarity} iff
\begin{equation} \tag{H\"older's inequality}
  s^2(x,y) \le s(x) \ s(y) .
\end{equation}

$$ - \sqrt {s(x) \ s(y)} \le s(x,y) \le \sqrt {s(x) \ s(y)}. $$

If $x, y \not \in C(s)$ then {\em cosine}
$$ \cos(x,y) \eqdef \frac {s(x,y)} {\sqrt{s(x) \ s(y)}} $$
is in $[-1,1]$.

$$ c \in C(s) \implies s(x,c) = 0. $$


\subsection{Induced distance and similarity}

The {\em $c$-centered comparison~$s_{d,c}$ induced} by a squared dissimilarity~$d^2$, where $c \in X$, is
$$ s_{d,c}(x,y) \eqdef \frac 1 2 \left(d^2(x,c) + d^2(y,c) - d^2(x,y) \right). $$
$$ s_{d,c}(x) = d^2(x,c). $$
$$ c \in C(s_{d,c}). $$

If~$d$ is a semidistance then~$s_{d,c}$ is a semisimilarity.

If~$d$ is a distance then~$s_{d,c}$ is a similarity.
\proof
{
    For any $x,y \in X$
    $$ d(x,c) \le d(x,y) + d(y,c). $$
    $$ d(x,c) - d(y,c) \le d(x,y). $$
    $$ \left(d(x,c) - d(y,c) \right)^2 \le d^2(x,y). $$
    $$ d^2(x,c) + d^2(y,c) - 2 \ d(x,c) \ d(y,c) \le d^2(x,y). $$
    $$ d^2(x,c) + d^2(y,c) - d^2(x,y) \le 2 \ d(x,c) \ d(y,c) . $$
    $$ 2 \ s_{d,c}(x,y) \le 2 \ \sqrt{s_{d,c}(x)} \ \sqrt{s_{d,c}(y)}. $$
    $$ s_{d,c}^2(x,y) \le s_{d,c}(x) \ s_{d,c}(y). $$
}

The squared dissimilarity~$d_s^2$ {\em induced} by a comparison~$s$ is
\begin{equation}\label{s2d}
  d_s^2(x,y) \eqdef s(x) + s(y) - 2 s(x,y).
\end{equation}
$$ d_s^2(x,x) = 0. $$
$$ c \in C(s) \implies d_s^2(x,c) = s(x). $$

If~$s$ is a similarity then~$d_s$ is a semidistance,
because
$$ 0 \le \left(\sqrt{s(x)} - \sqrt{s(y)} \right)^2 = s(x) + s(y) - 2 \sqrt{s(x) s(y)} \le s(x) + s(y) - 2 s(x,y) = d_s^2(x,y). $$

The {\em $c$-centered comparison~$s_c$ induced} by a comparison~$s$, where $c \in X$, is
\comm { comparison {\em translation}}
$$ s_c(x,y) \eqdef s(x,y) - s(x,c) - s(y,c) + s(c). $$
$$ s_c(x) = s(x) - 2 s(x,c) + s(c) = d_s^2(x,c). $$
$$ c \in C(s_c). $$
$$ c \in C(s) \implies s_c = s. $$

If~$s$ is a similarity then~$s_c$ is a semisimilarity.

If
$$\forall c \in X : s_c \text{ is a similarity}, $$
then~$d_s$ is a distance.
\proof
{
  For any $x,y,c \in X$
  $$ s_c^2(x,y) \le s_c(x) \ s_c(y). $$
  $$ 2 \ s_c(x,y) \le 2 \ \sqrt{s_c(x)} \ \sqrt{s_c(y)}. $$
  $$ d_s^2(x,c) + d_s^2(y,c) - d_s^2(x,y) \le 2 \ d_s(x,c) \ d_s(y,c) . $$
  $$ d_s^2(x,c) + d_s^2(y,c) - 2 \ d_s(x,c) \ d_s(y,c) \le d_s^2(x,y). $$
  $$ \left(d_s(x,c) - d_s(y,c) \right)^2 \le d_s^2(x,y). $$
  $$ d_s(x,c) - d_s(y,c) \le d_s(x,y). $$
  $$ d_s(x,c) \le d_s(x,y) + d_s(y,c). $$
}

\comm {group}

$$ \forall s, c : d_s = d_{s_c}. $$
$$ \forall d, c, z : s = s_{d,c} \implies s_z = s_{d,z} . $$

Example: $Y$ is a set, $X \eqdef 2^Y$, $d^2(A,B) \eqdef |(A \setminus B) \cup (B \setminus A)|$ is a squared distance,
and $s(A,B) \eqdef |A \cap B|$ is a similarity induced by~$d$ centered at~$\emptyset$.


\subsection{Finite set}

\comm{ $\to$ after "Square matrix" section}

Let $Y \subseteq X$ be a finite set.
$$ n \eqdef |Y|. $$

If~$s$ is a similarity, then
$$ \sum_{x,y \in Y} s(x,y) \le n \sum_{x \in Y} s(x), $$
because
$$ 0 \le \sum_{x,y \in Y} d_s^2(x,y) = \sum_{x,y \in Y} (s(x) + s(y) - 2 s(x,y)) = 2 n \sum_{x \in Y} s(x) - 2 \sum_{x,y \in Y} s(x,y). $$

Let~$s(Y)$ be the matrix of comparisons~$s$ on~$Y$.

The comparison~$s$ is a semisimilarity iff the determinants of all principal $k \times k$ submatrices of~$s(Y)$, where $k = 1$, are non-negative.
The comparison~$s$ is a similarity     iff the determinants of all principal $k \times k$ submatrices of~$s(Y)$, where $k \le 2$, are non-negative.

$$ s \succeq 0 \equiv \forall Y \subseteq X : |Y| \le \infty \implies s(Y) \succeq 0. $$

if $s \succeq 0$ then~$s$ is a similarity.
\comm {$s \succeq 0 \equivalent d_s \text{ is Euclidean}$ }

Let {\em center of~$Y$}
$$ C(Y) \eqdef \arg \min_{c \in X} \sum_{x \in Y} d^2(x,c). $$
$$ C(Y) = \arg \min_{c \in X} \sum_{x,y \in Y}s_{d,c} (x,y) = \arg \min_{c \in X} \sum_{x \in Y}s_{d,c} (x). $$

If there is $c \in X$, s.t.
$$ \forall x \in Y : s(x,c) = \frac 1 n \sum_{y \in Y} s(x,y) $$
and
$$ s(c) = \frac 1 n \sum_{x \in Y} s(c,x) = \frac 1 {n^2} \sum_{x,y \in Y} s(x,y), $$
then
$$ \sum_{x,y \in Y} s_c(x,y) = 0 $$
and
$$ \sum_{x \in Y} s_c(x) = \sum_{x \in Y} s(x) - \frac 1 n \sum_{x,y \in Y} s(x,y). $$

$$ s_c(Y) \succeq 0 \implies c \in C(Y). $$

$$ s_c(Y) = s(Y) - \mat A_n s(Y) - s(Y) \mat A_n + \mat A_n s(Y) \mat A_n = \mat C_n s(Y) \mat C_n. $$

\comm{matrix operations}



\section{Matrices}

\subsection{Vectors}

A {\em vector} is an element of~$\C^n$, which is a column with $n$ rows.

A vector has 2 interpretations:
\begin{enumerate}
  \item a sequence of~$n$ numbers;
  \item a {\em point} in an $n$-dimensional space.
\end{enumerate}


Let vectors be denoted by small bold non-slanted letters.

Let~$x_i$ be the $i$th element of the vector~$\ve x$.

Let~$\ve 1$ be a vector where each element is~$1$.
Let~$\ve 1_k$ be a vector where the first $k$ elements are~$1$ and the rest are~$0$.

Let~$\ve u_k$ be a vector where each element is~0, except for the~$k$th element which is~1.

$$ \ve x' \ve y \eqdef \sum_i x_i^* y_i. $$

The {\em average} of~$\ve x$ is $\bar {\ve x} \eqdef \frac 1 n \ve 1' \ve x$.

A vector $\ve x$ is {\em centered} iff $\bar{\ve x} = 0$.

$$ \bar {\ve x} + \bar {\ve y} = \overline {\ve x + \ve y}. $$

$$ \tilde {\ve x} \eqdef \ve x - \bar {\ve x} {\ve 1}. $$

$$\bar {\tilde {\ve x}} = 0. $$

$$ \tilde {\ve x} + \tilde {\ve y} = \widetilde {\ve x + \ve y}. $$

The {\em scatter} of~$\ve x$ is $|\ve x|^2 \eqdef \ve x' \ve x$.

$$ \var \ve x \eqdef \frac 1 n |\tilde {\ve x}|^2 = \frac 1 n |\ve x|^2 - \bar{\ve x}^2 $$
is the {\em variance} of~$\ve x$.

A vector $\ve x$ is {\em normalized} iff $|\ve x|^2 = 1$.

The {\em norm} of~$\ve x$ is $|\ve x| \eqdef \sqrt{|\ve x|^2}$.

The {\em length of the projection} of~$\ve y$ on~$\ve x$ is $\ve x' \ve y / |\ve x|$.

The {\em projection} of~$\ve y$ on~$\ve x$ is
\begin{equation} \label{vecProjection}
  \frac {\ve x'} {|\ve x|}  \ve y \ \frac {\ve x} {|\ve x|} = \frac {\ve x \ve x'} {|\ve x|^2} \ve y.
\end{equation}

$$ \cov (\ve x, \ve y) \eqdef \frac 1 n \tilde {\ve x}' \tilde {\ve y} = \frac 1 n \ve x' \ve y - \bar{\ve x} \bar{\ve y} $$
is the {\em covariance} between~$\ve x$ and~$\ve y$.
$$ \var (\ve x) = \cov(\ve x,\ve x). $$

$$ \cor (\ve x, \ve y) \eqdef \frac {\cov (\ve x, \ve y)} {\sqrt{\var \ve x} \ \sqrt{\var \ve y}} $$
is the {\em correlation} between~$\ve x$ and~$\ve y$.

The {\em Euclidean distance} $$d(\ve x, \ve y) \eqdef |\ve x - \ve y|.$$
If $x,y \in \{\R,\I\}^n$ then the Euclidean distance is a dissimilarity.
If $x,y \in \R^n$ then the Euclidean distance is a distance and has unidimensional between-sets.

The {\em Euclidean similarity} $$s(\ve x, \ve y) \eqdef \ve x' \ve y.$$
If $x,y \in \{\R,\I\}^n$ then the Euclidean similarity is a comparison.
If $x,y \in \R^n$ then the Euclidean similarity is a similarity centered at~$\ve 0$.

The sum of the squared differences of the elements of~$\ve x$ from~$a$
\begin{equation} \label{sqDiff}
|a \ve 1 - \ve x|^2 = a^2 \ve 1' \ve 1 + |\ve x|^2 - 2a\ve 1' \ve x = n (a - \bar {\ve x})^2 + n \ \var \ve x.
\end{equation}
If $\ve x \in \R^n$ and~$a \in \R$ then
\begin{equation} \label{centerL2}
  \arg \min_a |a \ve 1 - \ve x|^2 = \{\bar {\ve x} \}.
\end{equation}



\comm { Metrics $L_k$}

\comm { Are they distances ?}


\subsection{Rectangular matrices}

A {\em matrix} is an element of $\C^{n \times m}$, which is a table with $n$ rows and $m$ columns.

Let matrices be denoted by large non-slanted letters.

A vector $\ve x \in \C^n$ is equivalent to a matrix $\mat X \in \C^{n \times 1}$.

A matrix $\mat X$ is {\em square} iff $\mat X \in \C^{n \times n}$.

Let $\ve x_{i \cdot }$ be the $i$th row of matrix~$\mat X$,
$\ve x_{\cdot j}$ be the $j$th column of matrix~$\mat X$,
and $x_{ij}$ be the element of~$\mat X$ in row~$i$ and column~$j$.

If $\mat A \in \C^{n \times m_1}$ and $\mat B \in \C^{n \times m_2}$, then $[\mat B \ \mat C] \in \C^{n \times (m_1 + m_2)}$.
Similarly,
if $\mat A \in \C^{n_1 \times m}$ and $\mat B \in \C^{n_2 \times m}$, then $\left[ \begin{array}{c} \mat B \\ \mat C \end{array} \right] \in \C^{(n_1 + n_2) \times m}$.

For $\mat X \in \C^{n \times m}$, the matrix $\mat X' \in \C^{m \times n}$, where $x'_{ij} = x_{ji}$, is the {\em transpose of~$\mat X$}.

$$ \E \mat X \eqdef \bar {\mat X} \eqdef \frac 1 n \mat X' \ve 1. $$

$$ |\mat X|^2 \eqdef \mat X' \mat X. $$

Let $rank(\mat X)$ be the maximum number of linearly independent rows of~$\mat X$.
$$ rank(\mat X') = rank(\mat X). $$
\comm { \proof {}}
$$ rank(\mat X) \le \min \{n, m\}. $$
$$ rank(\mat A \mat B) \le \min \{rank (\mat A), rank (\mat B)\}. $$
$$ rank(\mat X \mat X') = rank (\mat X). $$
\begin{equation} \label{rank_nonempty_cells}
  |\{\langle i, j \rangle \mid x_{ij} \ne 0\}| \ge rank(\mat X).
\end{equation}
\comm{procedure to find all linear dependencies}

$\mat X \ve y$ has two interpretations:
\begin{enumerate}
  \item the lengths of the projections of~$\ve y$ on the rows of~$\mat X$ times the norms of the rows;
  \item the sum of the columns of~$\mat X$ weighted by~$\ve y$.
\end{enumerate}

\begin{equation}\label{sim2dataCentered}
\mat X \ve z = \ve 0 \equivalent \mat X' \mat X \ve z = \ve 0.
\end{equation}
\proof {
Sufficiency:
If $\mat X \ve z \ne \ve 0$ and $\mat X' \mat X \ve z = \ve 0$ then $0 \ne (\ve z' \mat X') (\mat X \ve z) = \ve z' (\mat X' \mat X \ve z) = \ve z' \ve 0 \ \#$.
}

$$\mat X \mat Y = \sum_i \ve x_{\cdot i} \ve y_{i \cdot}'. $$

The {\em generalized inverse} of~$\mat X \in \C^{n \times m}$ is the set of matrices
$$\mat X^- \eqdef \{\mat Y \in \C^{m \times n} : \mat X \mat Y \mat X = \mat X\}. $$
The symbol~$\mat X^-$ will be used in an expression~$f(\mat X^-)$ meaning any $\mat Y \in \mat X^-$,
presupposing $$\forall \mat Y \in \mat X^- : f(\mat Y) = const. $$



\subsection{Square matrices}

Let~$\mat X \in \C^{n \times n}$ .

A {\em principal} submatrix of~$\mat X$ is a matrix obtained by the removal of the rows and columns with the same indices from~$\mat X$.

$\mat X$ is {\em symmetric} iff $\mat X = \mat X'$.

$\mat X$ is {\em full rank} iff $rank(\mat X) = n$.

Full-rank matrices make up a group w.r.t.~multiplication.

If~$\mat B$ is full rank and $\ve x = \mat B \ve y$ then~$\ve y$ {\em represents~$\ve x$ in the space of~$\mat B$}.

If~$\mat X$ is full rank then $\mat X^{-1}$ is the {\em inverse} of~$\mat X$:
$$ \mat X \mat X^{-1} = \mat X^{-1} \mat X = \mat I, $$
$$ \mat X^- = \{\mat X^{-1}\} . $$

The {\em determinant} of matrix~$\mat X$, denoted by $\det \mat X$, is the {\em oriented volume} of~$\mat X$.
\comm { explain}
$$ rank (\mat X) < n \equivalent \det \mat X = 0. $$

$$ \mat X, \mat Y, \mat Z \in \C^{n \times n} \And \mat X = \mat Y \mat Z \implies \det \mat X = \det \mat Y \det \mat Z. $$

$\mat X$ is {\em skew-symmetric} iff $\mat X = - \mat X'$.

For any square~$\mat X$,
$\mat X_{sym} \eqdef (\mat X + \mat X') / 2$ is a symmetric matrix,
$\mat X_{skew} \eqdef (\mat X - \mat X') / 2$ is a skew-symmetric matrix,
$\mat X = \mat X_{sym} + \mat X_{skew}$
and $\forall \ve x : \ve x' \mat X \ve x = \ve x' \mat X_{sym} \ve x$.

$$ \mat X = - \mat X' \implies \forall i : x_{ii} = 0. $$

The {\em trace} of~$\mat X$
$$\trc \mat X \eqdef \sum_i x_{ii}.$$
If $\mat Y \in \C^{n \times m}$ and $\mat Z \in \C^{m \times n}$ then $\trc \mat Y \mat Z = \trc \mat Z \mat Y$.
In particular,
$$ \trc |\mat Y|^2 = \trc |\mat Y'|^2 = \sum_{ij} y^2_{ij}. $$

Let~$\ve x^d$ be the square matrix~$\mat X$ whose diagonal is vector~$\ve x$:
\begin{equation*}
x_{ij} \eqdef
  \begin{cases}
    x_i, &\text{ if } i = j\\
    0,   &\text{ else.}
  \end{cases}
\end{equation*}

$$ (\ve x^d)' = \ve x^d. $$

$$ \mat I \eqdef \ve 1^d. $$

$$ \ve x^{-d} \eqdef \left(\ve x^d \right)^{-1}. $$

$$ \ve x^{\sim d} \eqdef \ve y^d, $$
where
\begin{equation*}
  y_i \eqdef
  \begin{cases}
     0, &\text{ if } x_i = 0\\
     1/x_i,   &\text{ else.}
  \end{cases}
\end{equation*}

Let the matrix $\mat I_n^m \in \{0,1\}^{n \times m}$ be the first~$m$ columns of~$\mat I$.
And let the matrix $\mat I_n^{-m} \in \{0,1\}^{n \times m}$ be the last~$n-m$ columns of~$\mat I$.
$$ [\mat I_n^m \ \mat I_n^{-m}] = \mat I_n. $$

For a matrix~$\mat Y \in \C^{n \times m}$,
$\mat Y \mat I^k_m$ is the first $k$ columns of~$\mat Y$,
and
$(\mat I^k_n)' \mat Y$ is the first $k$ rows of~$\mat Y$.

$$\mat A \ve x^d \mat B = \sum_i x_i \ve a_{\cdot i} \ve b_{i \cdot}'. $$

A matrix~$\mat H \in \C^{n \times n}$ is {\em idempotent} iff
$$ \mat H^2 = \mat H. $$
The matrix $\mat I - \mat H$ is also idempotent.
$$ \mat H (\mat I - \mat H) = \mat 0. $$
$$ rank(\mat H) + rank(\mat I - \mat H) = n. $$
\comm{prove}
$$ \mat H = \mat I \equivalent rank(\mat H) = n. $$

\definition {Orthonormal matrix}
{If matrix~$\mat B \in \C^{n \times n}$, then $\mat B$ is {\em orthonormal} with column {\em basic vectors} iff $|\mat B|^2 = \mat I$.
}

If $\mat B$ is orthonormal then $\mat B$ is full rank, $\mat B^{-1} = \mat B'$ and $\mat B'$ is orthonormal.

Orthonormal matrices of the same size make up a group w.r.t.~multiplication.

A matrix~$\mat P$ is a {\em permutation matrix}
iff $\mat P \in \{0,1\}^{n \times n}$, $\ve 1' \mat P = \ve 1'$ and $\mat P \ve 1 = \ve 1$.
The matrix $\mat P \mat A$ is the permutation of the rows of the matrix~$\mat A$.
The matrix $\mat B \mat P$ is the permutation of the columns of the matrix~$\mat B$.
A permutation matrix is orthonormal.



\subsection{Semidefiniteness}

A symmetric matrix~$\mat X \in \R^{n \times n}$ is {\em positive semidefinite}, denoted by $\mat X \succeq 0$, iff
$$\forall \ve x \in \R^n: \ve x' \mat X \ve x \ge 0. $$
\comm {$\mat X \in \C^{n \times n}$}

A symmetric matrix~$\mat X \in \R^{n \times n}$ is {\em positive     definite}, denoted by $\mat X \succ   0$, iff
$$\forall \ve x \in \R^n: \ve x \ne \ve 0 \implies \ve x' \mat X \ve x > 0. $$
There are similar definitions for {\em negative semidefinite} and {\em negative definite} matrices.

$$ \mat X \succeq 0 \implies (\mat X \succ 0 \equivalent rank(\mat X) = n). $$

$$ \mat X \succeq 0 \implies \forall i : x_{ii} \ge 0. $$
$$ \mat X \succ 0 \implies \forall i : x_{ii} > 0. $$

$$ \forall \mat A \in \R^{n \times m}: |\mat A|^2 \succeq 0 \And |\mat A'|^2 \succeq 0 $$

\begin{equation} \tag{Cholesky decomposition}
\mat X \succeq 0 \equivalent \exists \mat U \in \R^{n \times n} : \mat U \text{ is upper-triangular} \And \mat X = |\mat U|^2.
\end{equation}
If $\mat X \succ 0$ then there is a unique Cholesky decomposition such that $\forall i : u_{ii} > 0$.
$$ \mat X \succ 0 \implies \mat X^{-1} = \mat U^{-1} (\mat U^{-1})'. $$
\begin{equation} \label{cholesky_rank}
rank(\mat X) = rank(\mat U).
\end{equation}
\begin{equation} \label{rank_nonempty_diagonal}
\mat X \succeq 0 \implies |\{i \mid x_{ii} > 0\}| \ge rank(\mat X),
\end{equation}
because~(\ref{cholesky_rank}), cf.~(\ref{rank_nonempty_cells}).

$$ \mat X \succeq 0 \implies \det \mat X \ge 0. $$
$$ \mat X \succeq 0 \equivalent (\forall \mat Y : \text{$\mat Y$ is a principal submatrix of $\mat X$} \implies \det \mat Y \ge 0). $$
\proof
{
    Sufficiency: If $\exists \mat Y$, s.t.~$\mat Y$ is a principal submatrix of $\mat X$ and $\det \mat Y < 0$,
    then $\mat Y = \mat B \ve \lambda^d \mat B'$ for some $\mat B$ and vector~$\ve \lambda$ containing $\lambda_i < 0$, see Section~\ref{eigen},
    and for $\ve y \eqdef \mat B \ve u_i$, $\ve y' \mat Y \ve y < 0$.

}



\subsection{Eigenvectors and eigenvalues} \label{eigen}

Let~$\mat X \in \C^{n \times n}$.

\definition {Eigenvectors and Eigenvalues}
{
For $\mat B \in \C^{n \times n}$ and $\ve \lambda \in \C^n$
such that
$$\mat X \mat B = \mat B \ve \lambda^d,$$
$$ rank(\mat B) = n, $$
$$ \forall i: |\ve b_{\cdot i}|^2 = 1 $$
and the elements of $\ve \lambda$ are ordered so that $\forall i : |\lambda_i| \ge |\lambda_{i+1}| \And (|\lambda_i| > |\lambda_{i+1}| \Or \phi_i > \phi_j)$, where $\lambda_i = |\lambda_i| e^{i \phi_i}$,
the columns of~$\mat B$ are the {\em eigenvectors} of~$\mat X$,
and the elements of~$\ve \lambda$ are the {\em eigenvalues} of~$\mat X$.

This is denoted by $\mat B \in \mathfrak B(\mat X)$ and $\ve \lambda = \ve \lambda(\mat X)$.
}

$$ \ve b \text{ is an eigenvector of } \mat X \implies \forall y \in \R : e^{i y} \ve b \text{ is an eigenvector of } \mat X. $$
\comm {$\ve b$ or $- \ve b$, normalization of $\ve b$ ??}

The elements of $\ve \lambda(\mat X)$ are the roots of the equation
$$ \det(\mat X - \lambda \mat I) = 0. $$

The vector~$\ve \lambda(\mat X)$ is unique for~$\mat X$.

$$ \ve \lambda(\mat X) = \ve \lambda(\mat X'). $$

The set~$\mathfrak B(\mat X)$ may contain more than one element.
For example, if for $a \le i \le b$ the values $\lambda_i$ are the same
then the eigenvectors $\ve b_{\cdot i}$ can be replaced by any basis of the subspace spanned by these eigenvectors.

$$ \forall i: i > rank(\mat X) \equivalent \lambda_i = 0. $$
$$ rank(\mat X) = |\{i \mid \lambda_i \ne 0\}|. $$

\begin{equation} \label{mat2eigen} \tag{eigenvalue decomposition}
  \mat X = \mat B \ve \lambda^d \mat B^{-1}.
\end{equation}

$$ \mat B^{-1} \mat X \mat B = \ve \lambda^d. $$
$$ \trc \mat X = \trc \mat B \ve \lambda^d \mat B^{-1} = \trc \mat B^{-1} \mat B \ve \lambda^d = \trc \ve \lambda^d = \ve 1' \ve \lambda. $$
$$ \det(\mat X) = \det(\ve \lambda^d) = \prod_i \lambda_i. $$


\subsubsection {Matrix transformations}
If~$\mat X$ is full rank or $n \ge 0$ then
$$ \mathfrak B(\mat X^n)  = \mathfrak B(\mat X) \text{ and } \ve \lambda^d(\mat X^n) = \left(\ve \lambda^d(\mat X) \right)^n, $$
because
$$ \mat X^n = \mat B (\ve \lambda^d)^n \mat B^{-1}. $$

\bigskip
In particular, if $\mat X$ is full rank then
$$\mathfrak B(\mat X^{-1}) = \mathfrak B(\mat X) \text{ and } \ve \lambda^d(\mat X^{-1}) = \ve \lambda^{-d} (\mat X) $$
and
$$ \mat X^{-1} = \mat B \ve \lambda^{-d} \mat B^{-1}. $$

\bigskip
$$ \mathfrak B(\mat X') = \{(\mat B^{-1})' \mid \mat B \in \mathfrak B(\mat X)\} \text{ and } \ve \lambda(\mat X') = \ve \lambda(\mat X), $$
because
$\mat X = \mat B \ve \lambda^d \mat B^{-1}$,
$\mat B^{-1} \mat X = \ve \lambda^d \mat B^{-1}$
and
$\mat X' (\mat B^{-1})' = (\mat B^{-1})' \ve \lambda^d$.

\bigskip
If~$\mat P$ is full rank then
$$ \mathfrak B(\mat P \mat X \mat P^{-1}) = \{\mat P \mat B \mid \mat B \in \mathfrak B(\mat X)\} \text{ and } \ve \lambda(\mat P \mat X \mat P^{-1}) = \ve \lambda(\mat X), $$
because $\mat P \mat X \mat P^{-1} \mat P \mat B = \mat P \mat B \ve \lambda^d$.

\bigskip
If $\mat X \succ 0$ with a Choleski decomposition $\mat X = \mat U' \mat U$,
then
\begin{equation} \label{eigenPdY}
  \forall \mat Y \in \C^{n \times n} :
    \mathfrak B \left((\mat U')^{-1} \mat Y \mat U^{-1} \right) \subset \{\mat U \mat B \ve \mu^d: \mat B \in \mathfrak B(\mat X^{-1} \mat Y), \ve \mu \in \R_{++}^n\}
    \And \ve \lambda(\mat X^{-1} \mat Y) = \ve \lambda \left( (\mat U')^{-1} \mat Y \mat U^{-1} \right).
\end{equation}
\proof
{
Let $\mat B \in \mathfrak B(\mat X^{-1} \mat Y)$ and $\ve \lambda \eqdef \ve \lambda(\mat X^{-1} \mat Y)$,
then
$$ \mat X^{-1} \mat Y \mat B = \mat B \ve \lambda^d. $$
$$ \mat Y \mat B = \mat X \mat B \ve \lambda^d = \mat U' \mat U \mat B \ve \lambda^d. $$
The matrix~$\mat U$ is full rank.
$$ (\mat U')^{-1} \mat Y \mat U^{-1} \ \mat U \mat B = \mat U \mat B \ \ve \lambda^d. $$
}

\subsubsection {Spectral decomposition}
For $\mat C \eqdef \mat B^{-1}$,
\begin{equation*} \tag{spectral decomposition}
\mat X = \sum_i \lambda_i \ve b_{\cdot i} \ve c_{i \cdot}'.
\end{equation*}

If
$$\mat Y \eqdef \mat X - \lambda_i \ve b_{\cdot i} \ve c_{i \cdot}'$$
then
$$ \mathfrak B(\mat Y) = \mathfrak B(\mat X) \text{ and } \ve \lambda'(\mat Y) = [\lambda_1 \dots \lambda_{i-1} \ 0 \ \lambda_{i+1} \dots \lambda_n]. $$

$$ \sum_{i:\lambda_i > 0} \lambda_i^{-1} \ve b_{\cdot i} \ve c_{i \cdot}' \in \mat X^-. $$

$$ \trc \mat X^n = \trc (\ve \lambda^d)^n. $$

If $\ve y \in \C^n$ is a random vector and $\ve z \eqdef \mat B^{-1} \ve y$, then $\mat X^n \ve y = \mat B (\ve \lambda^d)^n \ve z$ and
$$ \exists k \ne 0 \text{ with probability 1, s.t. } \lim_{n \to \infty} \mat X^n \ve y = k \ve b, $$
where $\ve b$ is a linear combination of the columns of~$\mat B$ having the maximum eigenvalues by absolute value.


\subsection{Linear space}

The {\em linear space spanned by the columns of~$\mat A \in \C^{n \times m}$}
$$ Sp(\mat A) \eqdef \{\mat A \ve x \mid \ve x \in \C^m\}. $$
$$ Sp(\mat A \mat B) \subseteq Sp(\mat A). $$
$$ rank(\mat A) = \min \{r \mid Sp(\mat A) = Sp(\mat B) \And \mat B \in \C^{n \times r} \}. $$

A matrix~$\mat P$ is an {\em orthogonal projector} iff
$$ \forall \ve x : (\mat P \ve x)' (\mat I - \mat P) \ve x = 0. $$

A matrix~$\mat P$ is an orthogonal projector iff $\mat P$~is symmetric and $\mat P^2 = \mat P$.

A matrix~$\mat P$ is an orthogonal projector iff $\mat I - \mat P$ is an orthogonal projector.

If a matrix~$\mat P$ is an orthogonal projector
then
$$ \mat P \succeq 0 $$
and
$$ \ve \lambda(\mat P) = [\ve 1_r' \ \ve 0']', $$
where $r \eqdef rank(\mat P)$.

If~$\mat B \in \C^{m \times m}$ is orthonormal,
then the orthogonal projector onto $\mat B \mat I^r_m$ is $\mat B \mat I^r_m \mat B'$.

Suppose $\mat A \in \R^{n \times m}$ and $\ve x \in \R^m$.
Then
$$ \min_{y \in Sp(\mat A)} |\ve x - \ve y| = \min_{\ve u} |\ve x - \mat A \ve u|^2. $$
$$ \frac d {d \ve u} |\ve x - \mat A \ve u|^2 = 0. $$
$$ \mat A' \mat A \ve u = \mat A' \ve x. $$
$$ \ve y = \mat A (\mat A' \mat A)^{-1} \mat A' \ve x. $$
$$ \mat P = \mat A (\mat A' \mat A)^{-1} \mat A'. $$
The matrix~$\mat P$ is an orthogonal projector onto $Sp(\mat A)$.


\subsection {Linear model of data reduction}
\comm { position in text}
$$ \mat Y = \mat F \mat A + \mat \varepsilon, $$
where $\mat F$ is the {\em factor} matrix,
$\mat A$ is the {\em loading} matrix,
and $\mat \varepsilon$ is the {\em residual} matrix.

$$ \mat F \mat A = \sum_k \ve f_{\cdot k} \ve a_{k \cdot}'. $$

If the value $\trc \mat \varepsilon' \mat \varepsilon$ is minimum then for all~$k$
$$\mat Y \ve a_{k \cdot} = |\ve a_{k \cdot}|^2 \ve f_{\cdot k}, $$
$$\mat Y' \ve f_{\cdot k} = |\ve f_{\cdot k}|^2 \ve a_{k \cdot}.$$
\comm {\proof {The derivative = 0 for fixed $\ve f$ or $\ve a$}}

The vectors~$\ve a_{k \cdot}$ and~$\ve f_{\cdot k}$ can be found by alternating optimization:
$$ \mat Y' \mat Y \ve a_{k \cdot} = \lambda_k \ve a_{k \cdot}, $$
$$ \mat Y \mat Y' \ve f_{\cdot k} = \lambda_k \ve f_{\cdot k}, $$
where
$$ \lambda_k \eqdef |\ve a_{k \cdot}|^2 |\ve f_{\cdot k}|^2. $$

For all~$k$
the vectors $\ve a_{k \cdot}/|\ve a_{k \cdot}|$ are the eigenvectors of $\mat Y' \mat Y$
and
the vectors $\ve f_{\cdot k}/|\ve f_{\cdot k}|$ are the eigenvectors of $\mat Y \mat Y'$
with the same eigenvalues~$\lambda_k$.
\comm { Consequence for MDS vs. PC}


\subsection{Symmetric matrices}

Let $\mat X \in \C^{n \times n}$.

\begin{align*}
  \mat X \text{ is symmetric} \implies  & \ \exists \mat B \in \mathfrak B(\mat X) : |\mat B|^2 = \mat I \And \mat B \in \R^{n \times n}, \\
  \mat X \text{ is symmetric} \impliedby& \ \exists \mat B \in \mathfrak B(\mat X) : |\mat B|^2 = \mat I.
\end{align*}
\begin{align*}
 \mat X \text{ is symmetric} \implies& \ \lambda(\mat X) \in \R^n, \\
 \mat X \succeq 0 \implies& \ \lambda(\mat X) \in \R_+^n, \\
 \mat X \succ   0 \implies& \ \lambda(\mat X) \in \R_{++}^n.
\end{align*}

\proof
{
  If~$\mat X$ is symmetric and $\ve x$ and $\ve y$ are different eigenvectors of~$\mat X$ with eigenvalues~$\lambda$ and~$\mu$ respectfully
  then
  $$ \ve y' \mat X \ve x = \ve y' \lambda \ve x = \mu \ve y' \ve x, $$
  $$ \ve y' \ve x \ (\lambda - \mu) = 0. $$
  If $\ve y' \ve x \ne 0$ then $\lambda = \mu$ and all eigenvectors with eigenvalue~$\lambda$ make up a linear space, and these eigenvectors can be replaced by any orthonormal basis of this space.
  Therefore, $|\mat B|^2 = \mat I$.

  If~$\mat X$ is symmetric and
  $$ \mat X (\ve x + i \ve y) = (\lambda + i \mu) (\ve x + i \ve y), $$
  where $\ve x, \ve y \in \R^n$,
  $\neg (\ve x = \ve 0 \And \ve y = \ve 0)$ and $\lambda, \mu \in \R$,
  then
  $$ \mat X \ve x = \lambda \ve x - \mu \ve y, $$
  $$ \mat X \ve y = \lambda \ve y + \mu \ve x, $$
  $$ \ve y' \mat X \ve x = \lambda \ve y' \ve x - \mu \ve y' \ve y = (\ve x' \mat X \ve y)' = \lambda \ve y' \ve x + \mu \ve x' \ve x, $$
  $$ \mu (|\ve x|^2 + |\ve y|^2) = 0 $$
  and
  $$ \mu = 0, $$
  therefore, $\mat B \in \R^{n \times n}$.
}

Let an eigenvector matrix $\mat B \in \mathfrak B (\mat X)$ be orthonormal and $\ve \lambda \eqdef \ve \lambda (\mat X)$.

$$  \mat X = \mat B \ve \lambda^d \mat B'. $$

$$ \mat X' \mat B = \mat B \ve \lambda^d. $$

$$ \mat X \succeq 0 \equivalent \forall i : \lambda_i \ge 0. $$
$$ \mat X \preceq 0 \equivalent \forall i : \lambda_i \le 0. $$

$$ \mat X \succeq 0 \equivalent \mat X^{-1} \succeq 0. $$
$$ \mat X \preceq 0 \equivalent \mat X^{-1} \preceq 0. $$

Let $\mat C \eqdef \mat B \sqrt{\ve \lambda^d}$,
then $\mat X = \mat C \mat C'$
and
\begin{equation*}
\mat X \ve y = \ve 0 \equivalent \mat C' \ve y = \ve 0.
\end{equation*}

$$ \mat A \text{ is symmetric } \implies \frac d {d \ve x} \ve x' \mat A \ve x = 2 \mat A \ve x. $$


\subsubsection{Canonical analysis}  \label{CanonicalAnalysis}
\comm{name}
A {\em quadric} is
$$ \{\ve x \mid \ve x' \mat X \ve x = c\}. $$
If~$\mat X$ is full rank then a quadric is an {\em ellipsoid}.

If~$\mat X \succ 0$ and $\mat Y \in \C^{n \times n}$,
then the local extrema of $\{\ve x' \mat Y \ve x \mid \ve x' \mat X \ve x = 1\}$
are the columns of $\mathfrak B(\mat X^{-1} \mat Y)$.
\proof{
$$ \ve 0 = \frac d {d \ve x} (\ve x' \mat Y \ve x - \lambda (\ve x' \mat X \ve x - 1)) = 2 \mat Y \ve x - 2 \lambda \mat X \ve x, $$
where $\lambda$ is a Lagrange multiplier.
$$ \mat Y \ve x = \lambda \mat X \ve x. $$
}
\comm {extrema = maxima}

Let $\mat X = \mat U' \mat U$ be a Cholesky decomposition,
then
$$\ve \lambda(\mat X^{-1} \mat Y) = \ve \lambda \left((\mat U')^{-1} \mat Y \mat U^{-1} \right)$$
by~(\ref{eigenPdY}).

\bigskip
If~$\mat Y$ is symmetric then $(\mat U')^{-1} \mat Y \mat U^{-1}$ is symmetric
and
$$\ve \lambda(\mat X^{-1} \mat Y) \in \R^n,$$
$$\mat Y \succeq 0 \implies \ve \lambda(\mat X^{-1} \mat Y) \in \R_+^n.$$

Let  $\mat B \in \mathfrak B(\mat X^{-1} \mat Y)$ and $\mat C \in \mathfrak B \left((\mat U')^{-1} \mat Y \mat U^{-1} \right)$,
then $\mat C = \mat U \mat B \ve \mu^d$, for some $\ve \mu \in \R_{++}^n$, by~(\ref{eigenPdY}) and
for $$ \mat R \eqdef \mat U^{-1} \mat C = \mat B \ve \mu^d$$
\begin{equation} \label{canonicalWithin}
  \mat I = \mat C' \mat C = \ve \mu^d \mat B' \mat U' \mat U \mat B \ve \mu^d = \mat R' \mat X \mat R,
\end{equation}
\begin{equation} \label{canonicalBetween}
 \ve \lambda^d(\mat X^{-1} \mat Y) = \ve \lambda^d((\mat U')^{-1} \mat Y \mat U^{-1}) = \mat C' (\mat U')^{-1} \mat Y \mat U^{-1} \mat C
  = \mat R' \mat Y \mat R.
\end{equation}



\subsection{Singular value decomposition}

For~$\mat X \in \C^{n \times m}$, where $n \ge m$,
let
$r \eqdef rank(\mat X)$,
$\mat B \in \mathfrak B\left(|\mat X|^2\right) \cap \R^{m \times m}$,    $\ve \mu \eqdef \lambda\left(|\mat X|^2\right)$,
$\mat C \in \mathfrak B\left(|\mat X'|^2\right) \cap \R^{n \times n}$ and $\ve \nu \eqdef \lambda\left(|\mat X'|^2\right)$,
then
$\ve \mu \in \R^m$,
$\ve \nu \in \R^n$
and
$$ \mat X \mat X' (\mat X \mat B) = (\mat X \mat B) \ve \mu^d, $$
therefore, $\ve \mu' = [\ve \lambda' \ \ve 0']$ and $\ve \nu' = [\ve \lambda' \ \ve 0']$, where $\ve \lambda \in \R^r$ and $\forall i : \lambda_i \ne 0$,
and
\begin{equation} \label{rectangular_eigenvectors}
\exists \mat C \in \mathfrak B(\mat X \mat X') : \mat C \ \mat I^r_n = \mat X \mat B \ \mat I^r_n \sqrt{\ve \lambda^{-d}},
\end{equation}
because $\mat B' \mat X' \mat X \mat B = \mat B' \mat B \ve \mu^d = \ve \mu^d$ and $(\mat I^r_n)' \mat C' \mat C \ \mat I^r_n = \mat I_r$.

Similarly,
$$ \exists \mat B \in \mathfrak B(\mat X' \mat X) : \mat B \ \mat I^r_n = \mat X' \mat C \ \mat I^r_n \sqrt{\ve \lambda^{-d}}. $$

The elements of the vector~$\ve \lambda$ are {\em singular values}.

$$ (\mat C \ \mat I^r_n)' \ \mat X \ (\mat B \ \mat I^r_n) = (\mat I^r_n)' \mat C ' \mat C \ \mat I^r_n \sqrt{\ve \lambda^d} = \sqrt{\ve \lambda^d}. $$
$$ \mat C' \mat X \mat B = \mat M, $$
where
$$ \mat M \eqdef \left[\sqrt{\ve \mu^d} \ \mat 0 \right]'. $$

\begin{equation*} \tag{singular-value decomposition}
  \mat X = \mat C \mat M \mat B',
\end{equation*}
cf.~eigenvalue decomposition.
$$ \mat X = \mat C \ \mat I^r_n \ \sqrt{\ve \lambda^d} \ (\mat B \ \mat I^r_n)'. $$
\comm{cf. canonical form, Gram-Schmidt orthogonalization}

$$ \mat C \ \mat I^r_n \ \sqrt{\ve \lambda^{\sim d}} \ (\mat B \ \mat I^r_n)' \in \mat X^-. $$

$$ \mat X \in \R^{n \times m} \equivalent \ve \lambda \in \R_{++}^m. $$



\subsection{Strain}

Let the {\em strain} between matrices~$\mat X, \mat Y \in \C^{n \times m}$ be
$$ d^2(\mat X,\mat Y) \eqdef \sum_i d^2(\ve y_{i \cdot}, \ve z_{i \cdot}), $$
which is the squared Euclidean distance between vectorized matrices.

$$ d^2(\mat X,\mat Y) = \sum_j d^2(\ve y_{\cdot j}, \ve z_{\cdot j}) = \sum_{ij} |y_{ij} - z_{ij}|^2 = \trc |\mat X - \mat Y|^2. $$

The strain minimization formulas in this section are referred to as the {\em Eckart-Young theorem}.


\subsubsection{Strain minimization for rectangular matrices}

Let $\mat X, \mat Y \in \C^{n \times m}$ and the singular value decomposition of~$\mat X$ be $\mat C \mat M \mat B'$,
then
$$ \mat X - \mat Y = \mat C \mat C' (\mat X - \mat Y) \mat B \mat B' = \mat C (\mat M - \mat C' \mat Y \mat B) \mat B' = \mat C (\mat M - \mat W) \mat B', $$
where
$$ \mat W \eqdef \mat C' \mat Y \mat B. $$
$$ d^2(\mat X,\mat Y) = \trc \mat B (\mat M - \mat W)' \mat C' \mat C (\mat M - \mat W) \mat B' = \trc |\mat M -\mat W|^2 . $$
$$ rank(\mat W) = rank(\mat Y). $$

Suppose $\mat X, \mat Y \in \R^{n \times m}$.
Then
$$ d^2(\mat X,\mat Y) \ge 0, $$
$$ \min_{rank(\mat Y) \le k} d^2 (\mat X, \mat Y) = \min_{rank(\mat W) \le k} \trc |\mat M -\mat W|^2. $$

$$ \arg \min_{rank(\mat W) \le k} \trc |\mat M -\mat W|^2 = \{[\mat M \mat I_m^k \ \mat 0]' \} . $$
\proof
{ Suppose the minimzation criterion is attained at $\mat W = \mat W^*$ and $\mat W^* \ne [\mat M \mat I_m^k \ \mat 0]'$,
then~$\mat W^*$ can be transformed to $[\mat M \mat I_m^k \ \mat 0]'$
by combinations of elementary row or column rank-preserving transformations
decreasing the minimization criterion.
}
$$ \min_{rank(\mat Y) \le k} d^2 (\mat X, \mat Y) = \sum_{i > k} \lambda_i. $$


\subsubsection{Strain minimization for square matrices}
Suppose $\mat X, \mat Y \in \C^{n \times n}$ and the eigenvalue decomposition of~$\mat X$ is $\mat B \ve \lambda^d \mat B^{-1}$
then,
by the reasoning analogous to the previous section,
$$ \mat W \eqdef \mat B^{-1} \mat Y \mat B, $$
$$ d^2(\mat X,\mat Y) = \trc |\ve \lambda^d -\mat W|^2. $$

Suppose $\mat X, \mat Y \in \R^{n \times n}$.
Then
$$ d^2(\mat X,\mat Y) \ge 0, $$
$$ \arg \min_{rank(\mat W) \le k} \trc |\ve \lambda^d -\mat W|^2 = \{[\ve \lambda^d \mat I_m^k \ \mat 0]' \}, $$
$$ \min_{rank(\mat Y) \le k} d^2 (\mat X, \mat Y) = \sum_{i > k} \lambda_i^2. $$

Suppose~$\mat X, \mat Y$ are symmetric.
Then
$$ \mat Y \succeq 0 \equivalent \mat W \succeq 0, $$
$$ \min_{\mat Y \succeq 0} d^2 (\mat X, \mat Y) = \min_{\mat W \succeq 0} \trc |\ve \lambda^d -\mat W|^2, $$
$$ \arg \min_{\mat W \succeq 0} \trc |\ve \lambda^d -\mat W|^2 = \{\max (\ve \lambda^d, \mat 0) \} $$
and
$$ \min_{\mat Y \succeq 0} d^2 (\mat X, \mat Y) = \sum_{\lambda_i < 0} \lambda_i^2. $$

The constraints $rank(\mat Y) \le k$ and $\mat Y \succeq 0$ can be combined.



\subsection{Matrix transformations}
Linear transformations of a matrix~$\mat X \in \C^{n \times p}$:
\begin{itemize}
\item If~$\ve x$ is a row of~$\mat X$ then $\ve x + \ve y$ is a {\em translation} of row~$\ve x$ by vector~$\ve y$.
\item $\mat X \mat A$ is a {\em projection} of~$\mat X$ onto the rows of~$\mat A$.
\item If~$\mat B$ is orthonormal then $\mat X \mat B$ is a {\em rotation} of~$\mat X$ by changing the basis to the columns of~$\mat B$.
\comm {a.k.a. orthogonal transformation}
\item If $\ve \lambda \ge \ve 0$ then $\mat X \ve \lambda^d$ is a {\em stretching} of~$\mat X$.
\item If~$\ve x$ is a row of~$\mat X$ then $- \ve x$ is a {\em reflexion} of row~$\ve x$.
\end{itemize}

If $\mat Y \succeq 0$ then
$$ \ve x' \mat Y =  \ve x' \ \mat B \ \ve \lambda^d(\mat Y) \ \mat B^{-1}, $$
where $\mat B \in \mathfrak B(\mat Y)$,
by~(\ref{mat2eigen}),
i.e., $\ve x' \mat Y$ is the stretching of~$\ve x$ in the changed basis~$\mat B$.
Due to the ordering of the elements of~$\ve \lambda(\mat Y)$, the vector~$\ve x$ is stretched more along the first basic vectors of~$\mat B$.

Let $\mat A_n \eqdef \frac 1 n \ve 1 \ve 1'$ be an {\em averaging matrix}.
$\mat A^2_n = \mat A_n$.
$ \ve 1' \mat A_n = \ve 1'$.

$\mat X$ is {\em centered} iff $\ve 1' \mat X = \ve 0$.

Let $\mat C_n \eqdef \mat I_n - \mat A_n$ be a {\em centering matrix}, then the {\em centering} operation
$$ \bar {\mat X} \eqdef \mat C_n \mat X. $$
The matrix $\mat C_n$ has $n-1$ eigenvalues~1 and one eigenvalue~0.

$$ \ve 1' \bar {\mat X} = \ve 0. $$
$$ \bar {\bar {\mat X}} = \bar {\mat X}. $$

$$ \ve 1 \ve 1' \mat C_n = \mat 0. $$
$$ (\mat X + k \ve 1 \ve 1') \ \mat C_n = \mat X \mat C_n. $$

If~$\mat X$ is a square matrix with~$n$ rows then the {\em double centering} operation
$$ \mat C_n \mat X \mat C_n. $$

$$ \cov(\mat X, \mat Y) \eqdef \frac 1 n \mat X' \mat Y. $$
If matrices~$\bar {\mat X} = \bar {\mat Y} = \ve 0$ then $\cov(\mat X, \mat Y)$
is the MLE of the covariance matrix between the columns of~$\mat X$ and~$\mat Y$.

$$ \VC (\mat X) \eqdef \cov(\mat X, \mat X). $$
$$ \VC(\mat X) \succeq 0. $$

If $\bar {\mat X} = \ve 0$ then $\VC(\mat X)$ is the MLE of the variance-covariance matrix of the columns of~$\mat X$.

$$ \VC(\mat X \mat A) = \mat A' \VC (\mat X) \mat A. $$

$$ \mathfrak B (\mat X' \mat X) = \mathfrak B (\VC (\mat X)). $$


\section{Data}
Let {\em data} be a pair $(\mat X, \ve w)$,
where $\mat X \in \{\R \cup \nul\}^{n \times p}$ is the {\em data matrix} and $\ve w \in \R_+^n$ is the {\em multiplicity vector}.
Each row of~$\mat X$ is an {\em object} and each column of~$\mat X$ is an {\em attribute}.
The attribute~$\ve 1$ is an {\em intercept}.

The value $\nul \not \in \C$ is the {\em missing value}.

The objects of~$\mat X$ are points in the $p$-dimensional Euclidean space, which is the {\em space of objects}.
The attributes of~$\mat X$ are points in the $n$-dimensional Euclidean space, which is the {\em space of attributes}.

\subsection{Multiplicity}
The element~$w_i$ is the {\em multiplicity} of object~$\ve x_{i\cdot}$.
The data where $\ve w = \ve 1$ is {\em non-multiplied}.
If~$\ve w$ is omitted it is assumed that $\ve w = \ve 1$.

$$ w \eqdef \sum_i^n w_i. $$

For binary attributes, the multiplicity of a pair of objects~$(i,j)$ is $w_i w_j$.
\comm{}

The following are {\em multiplicity transformations} of data:
\begin{itemize}
\item $(\mat X, \ve w) \leftrightarrow (\mat P \mat X, \mat P \ve w)$, where $\mat P$ is a permutation matrix.
\item $\left(\left[ \begin{array}{c} \mat X \\ \mat X \end{array} \right], \left[ \begin{array}{c} \ve w_1 \\ \ve w_2 \end{array} \right] \right)
        \leftrightarrow (\mat X, \ve w_1 + \ve w_2)$,
  where $w_1, w_2 \in \R_+^n$.
\item $\ve w_0 = \ve 0 \implies
        \left(\left[ \begin{array}{c} \mat X_1 \\ \mat X_0 \\ \mat X_2 \end{array} \right], \left[ \begin{array}{l} \ve w_1 \\ \ve w_0 \\ \ve w_2 \end{array} \right] \right)
          \leftrightarrow \left(\left[ \begin{array}{c} \mat X_1 \\ \mat X_2 \end{array} \right], \left[ \begin{array}{c} \ve w_1 \\ \ve w_2 \end{array} \right] \right)$,
  where $X_i \in \R^{n_i \times p}$ and $w_i \in \R_+^{n_i}$.
\item $(\mat X, \ve w) \leftrightarrow (\mat X, k \ve w)$, where $k \in \R_{++}$.
\end{itemize}

The multiplicity transformations of data make up an equivalence relation on data.

Any equivalence class of data w.r.t.~the multiplicity transformations contains non-multiplied data,
because any data can be transformed by the multiplicity transformations into non-multiplied data.

A {\em data analysis procedure} is a function of data.

\comm{consistent analysis}
A data analysis procedure is {\em consistent} iff it is a function of the {\em averaging statistics}
$$ \frac {\sum_i^n f(\ve x_{i\cdot}) w_i} {\sum_i^n w_i}, $$
where~$f$ is an arbitrary function $\R^p \mapsto \R$.

A consistent data analysis procedure is invariant w.r.t.~the multiplicity transformations.

If each row~$i$ of the data matrix represents an i.i.d.~multivariate random variable with the same distribution as~$\ve X$
and has missing values distributed independently of~$\ve x_{i\cdot}$ and~$w_i$
then averaging statistics are consistent estimates of~$f(\ve X)$.

\comm { Is Fisher exact test a data analysis procedure }

A data analysis procedure has {\em quality criteria}.
\comm { To be minimized}

\comm { {\em relative} quality criteria increase with~$n$, converges. For small~$n$ they are biased.}

An algebraic expression on attributes has two interpretations: a {\em multiplied interpretation} and a {\em non-multiplied interpretation}.
Below the expressions on the  left-hand part of the page have the multiplied interpretation
and   the expressions on the right-hand part of the page have the non-multiplied interpretation:
\begin{align*}
\ve x' \ve y &= \ve x' \ve w^d \ve y, \\
n = \ve 1' \ve 1 &= \ve 1' \ve w^d \ve 1 = w, \\
\bar {\ve x} = \frac {\ve 1' \ve x} {\ve 1' \ve 1},& \\
\var \ve x = \frac {\ve x' \ve x} {\ve 1' \ve 1} - {\bar {\ve x}}^2,& \\
\mat X' \mat X &= \mat X' \ve w^d \mat X,
\end{align*}
etc.

In the sequel all expressions on attributes with no~$\ve w$ in them have the multiplied interpretation.

$$ n_w \eqdef |\{i : w_i > 0\}|. $$

\comm { Algebra: $model : data, parameters \to data, quality$}

\comm {$size \in parameters$. Model is consistent $\implies$ $\lim_{size \to \infty} model(size)$ converges.}

\comm { model can generate data}




\subsection {Euclidean distance and Euclidean similarity between objects}
Let $\mat X \in \R^{n \times p}$ a data matrix.

The functions $s(\ve x_{i \cdot}, \ve x_{j \cdot})$ and $d^2(\ve x_{i \cdot}, \ve x_{j \cdot})$ are invariant w.r.t.~basis change.
The function $d^2(\ve x_{i\cdot}, \ve x_{j\cdot})$ is invariant w.r.t.~translation.

If~$\ve x_{\cdot i}$ and~$\ve x_{\cdot j}$ are attributes which are centered and normalized
then
\begin{equation}\label{d2rho}
d^2(\ve x_{\cdot i}, \ve x_{\cdot j}) = 2 (1 - \cor(\ve x_{\cdot i}, \ve x_{\cdot j})).
\end{equation}

Let
$$ d_{ij} \eqdef d(\ve x_{\cdot i}, \ve x_{\cdot j}), $$
$$ s_{ij} \eqdef s(\ve x_{\cdot i}, \ve x_{\cdot j}). $$

Let $\mat D^2(\mat X)$ be the squared Euclidean distance matrix of~$\mat X$.
$$ \sum_{i,j} d^2_{ij} = \ve 1' \mat D^2(\mat X) \ve 1. $$
$$ \forall \ve x : \ve x' \ve 1 = 0 \implies \ve x' \mat D^2 \ve x \le 0. $$

If~$\bar {\mat X} = \ve 0$ then
\begin{equation}\label{sumD2}
\ve 1' \mat D^2(\mat X) \ve 1 = 2n \ \trc \mat X' \mat X  = 2n^2 \ \trc \VC (\mat X).
\end{equation}

Let $\mat S(\mat X) \eqdef \mat X \mat X'$ be the Euclidean similarity matrix of~$\mat X$.
$$\mat S(\mat X) \succeq 0. $$
\begin{equation}\label{trc_S_VC}
  \trc \mat S(\mat X) = n \ \trc \VC(\mat X).
\end{equation}

If~$\bar {\mat X} = \ve 0$ then
$$ \mat S(\mat X) \ve 1 = \ve 0. $$

Let~$\mat S \eqdef \mat S(\mat X)$ and $\mat D^2 \eqdef \mat D^2 (\mat X)$ for an unknown~$\mat X$.

$$ \mat S(\bar {\mat X}) =  \mat C_n {\mat S} (\mat X)  \mat C_n. $$

The matrix $\mat D^2 \eqdef \mat D^2(\mat X)$ can be derived from $\mat S \eqdef \mat S (\mat X)$ by~(\ref{s2d}).
This procedure will be referred to as~$\mat D_{\mat S}(\mat S)$.
$$ \frac {\ve 1' \mat D_{\mat S}^2(\mat S) \ve 1} {2n} = \trc \mat S - \frac {\ve 1' \mat S \ve 1} n = \trc \mat C_n \mat S \mat C_n. $$

Conversely, the matrix $\mat S \eqdef \mat S (\mat X)$ can be derived from $\mat D^2 \eqdef \mat D^2(\mat X)$,
but only if $\bar {\mat X} = \ve 0$.
This procedure will be referred to as~$\mat S_{\mat D^2}(\mat D^2)$.
Since
$$ \forall i : \sum_j s_{ij} = 0, $$
$$ \forall i : \sum_j d^2_{ij} = n s_i + \sum_j s_j, $$
$$ \sum_{i,j} d^2_{ij} = 2 n \sum_i s_i, $$
let
$$ a \eqdef \frac {\sum_{i,j} d^2_{ij}} {2n}, $$
then
$$ s_i = \frac {\sum_j d^2_{ij} - a} n, $$
$$ s_{ij} = \frac {s_i + s_j - d^2_{ij}} 2. $$
The above two equations are equivalent to
$$ \mat S = - \frac 1 2 \mat C_n \mat D^2 \mat C_n. $$
\comm {If $\mat D^2$ is an arbitrary symmetric matrix then the $\mat S$ obtained by $\mat C_n$ is double-centered, but obtained via $a$ is not.}

$$ a = \sum_i s_i, $$
$$ \left(\forall i : s_i = \frac a n \right) \implies s_{ij} = \frac a n - \frac {d^2_{ij}} 2. $$


\subsection{Mahalanobis distance}
If a data matrix~$\mat X$ is full rank and $\bar {\mat X} = \ve 0$,
then the $i$th diagonal element of
\begin{equation} \label{mahalanobis}
\mat X \ \VC^{-1}(\mat X) \ \mat X'
\end{equation}
is the squared {\em Mahalanobis distance} of object~$i$.

If the objects of~$\mat X \in \R^{n \times p}$ have a multivariate normal distribution, then their squared Mahalanobis distance has the $\chi^2(p)$ distribution.


\subsection{Boolean data}
Let~$\mat X$ be a data matrix with Boolean attributes.

The {\em Hamming distance}
$$ d_h(\ve x_{i\cdot}, \ve x_{j\cdot}) \eqdef \sum_k |x_{ik} - x_{jk}|. $$
$$ d_h(\ve x_{i\cdot}, \ve x_{j\cdot}) = d^2 (\ve x_{i\cdot}, \ve x_{j\cdot}). $$
\comm { = distance in $L_1$, }
\comm { = squared distance in $L_2$}

If~$\mat X$ is random, then let the data matrix~$\mat Y$ be defined as
$$ y_{ik} \eqdef \P (X_{ik} = 1), $$
then
$$ \E |X_{ik} - X_{jk}|
    =   y_{ik} (1 - y_{jk})
      + y_{jk} (1 - y_{ik})
    =   y_{ik}  + y_{jk} - 2 y_{ik} y_{jk}.
$$

Compare with
$$ (y_{ik} - y_{jk})^2 = y^2_{ik} + y^2_{jk} - 2 y_{ik} y_{jk}. $$

Since $0 \le y_{ik} \le 1$,
then $y_{ik} \ge y^2_{ik}$
and
$$ \E d_h (\ve X_i, \ve X_j) \ge d^2 (\ve y_i, \ve y_j). $$

\comm{
Assume that~$\mat X[i,k]$ are independent for different~$i$.

Let~$Z_k \sim Bernoulli(\mat X[I,k])$, where $I \sim Uniform(1,n)$,
then
$$ \E Z_k = \frac 1 n \sum_{i=1}^n \mat Y[i,k], $$
\begin{equation}
\begin{split}
\var Z_k &= \frac 1 n \sum_{i=1}^n ((1-\E Z_k)^2 \mat Y[i,k] + (E Z_k)^2 (1 - \mat Y[i,k])) \\
         &= \frac 1 n \sum_{i=1}^n ( \mat Y[i,k] - 2 (\E Z_k) \mat Y[i,k] + (E Z_k)^2) \\
         &= \E Z_k - (\E Z_k)^2.
\end{split}
\end{equation}

Let data matrix~$\mat Z$ be defined as
$$ \mat Z[i,k] \eqdef \frac {\mat X[i,k] - \E Z_k} {\sqrt {\var Z_k}} $$
be the {\em normalized random Boolean data matrix}~$\mat X$,
then
$$ \E d_h (\ve z_{ik}, \ve z_{jk}) = \frac {\E d_h (\ve x_{ik}, \ve x_{jk})} {\var Z_k}. $$
}


\section{Construction of attributes}

\subsection{Principal component analysis}
Let~$\mat X \in \R^{n \times p}$ be a data matrix, such that $\bar {\mat X} = \ve 0$.

Let $q \le p$.

$$ \mathcal B_q \eqdef \{\mat B_q \in \R^{p \times q} : \mat B_q' \mat B_q = \mat I \}, $$
$$ \mat I^q_p \in \mathcal B_q. $$

Let $\mat B_p \in \mathcal B_p$ and $\mat B_q \in \mathcal B_q$, such that $\mat B_q = \mat B_p \mat I^q_p$.
Then in the data matrix $\mat X \mat B_q$ the dimensionality is reduced from~$p$ to~$q$.
\comm {{\em best} dimensionality reduction, also MDS, see Eckart-Young theorem}

For any $i,j \le n$:
$$ d^2(\ve x_{i\cdot}, \ve x_{j\cdot}) = d^2(\ve x_{i\cdot}' \mat B_p, \ve x_{j\cdot}' \mat B_p) \ge d^2(\ve x_{i\cdot}' \mat B_q, \ve x_{j\cdot}' \mat B_q). $$
The {\em lost squared distance} between objects $\ve x_{i\cdot}$ and~$\ve x_{j\cdot}$
$$ \epsilon^2_{ij} \eqdef d^2(\ve x_{i\cdot}, \ve x_{j\cdot}) - d^2(\ve x_{i\cdot}' \mat B_q, \ve x_{j\cdot}' \mat B_q) \ge 0. $$
The {\em total lost squared distance}
$$ \epsilon^2(\mat X, \mat B_q) \eqdef \sum_{i,j} \epsilon^2_{ij}. $$
$$ \epsilon^2(\mat X, \mat B_q) = \sum_{i,j} d^2(\ve x_{i\cdot}, \ve x_{j\cdot}) - \sum_{i,j} d^2(\ve x_{i\cdot}' \mat B_q, \ve x_{j\cdot}' \mat B_q). $$

An element of the set
$$ \mathfrak R_q(\mat X) \eqdef \arg \min_{\mat B_q \in \mathcal B_q} \epsilon^2(\mat X, \mat B_q) $$
is a {\em basis of the first $q$~rotated principal components} of~$\mat X$.

It can be rotated:
$$ \forall \mat B_q \in \mathfrak R_q(\mat X) \ \forall \mat C \in \R^{q \times q}:
  (\mat C' \mat C = \mat I \implies \mat B_q \mat C \in \mathfrak R_q(\mat X)).
$$

An element of the set
$$ \mathfrak B_q(\mat X) \eqdef \{\mat B \mat I^q_p : \mat B \in \mathfrak B(\VC(\mat X)) \} $$
is a {\em basis of the first $q$~principal components} of~$\mat X$.

$$ \ve \lambda \eqdef \lambda(\VC(\mat X)). $$

$$ \mathfrak B_q(\mat X) \subseteq \mathfrak R_q(\mat X). $$
\proof
{
By~(\ref{sumD2}),
$$ \mathfrak R_q(\mat X) = \arg \max_{\mat B_q \in \mathcal B_q} \trc \mat B_q' \VC (\mat X) \mat B_q. $$
$$ \mathfrak R_q(\mat X) =
  \{ \mat B \mat A_q : \mat B \in \mathfrak B(\VC(\mat X)),
                       \mat A_q \in \arg \max_{\mat A_q \in \mathcal B_q} \trc \mat A_q' \ve \lambda^d \mat A_q
  \}.
$$

\comm {Prove that $\mat I^q_p \in \arg \max_{\mat A_q \in \mathcal B_q} \trc \mat A_q' \ve \lambda^d \mat A_q$ }

True for $q = 1$.
}

If $\mat B_q \in \mathfrak B_q(\mat X)$ then the data matrix
$$\PC_q(\mat X) \eqdef \mat X \mat B_q$$
is the {\em first $q$ principal components} of~$\mat X$.

If the columns of~$\mat B_q$ are interpreted as unit-length orthogonal objects of~$\mat X$,
then the data matrix $\PC_q(\mat X)$ is the matrix of similarities between~$\mat X$ and~$\mat B_q$.

The matrix $\mat B_q$ depends on a linear transformation of~$\mat X$.
The first principal components tend to correlate with the attributes having largest scatter.
The matrix~$\PC_q(\mat X)$ does not depend on measurement scales of attributes of~$\mat X$ if the attributes are normalized.

$$ \E \PC_q(\mat X) = \ve 0. $$

$$ \ve \lambda_q(\mat X) \eqdef (\mat I^q_p)' \ve \lambda. $$

$$ \VC (\PC_q(\mat X)) = \VC (\mat X \mat B_q)
                = \mat B_q' \VC (\mat X) \mat B_q
                = \mat B_q' \mat B_q \ve \lambda_q^d(\mat X)
                = \ve \lambda_q^d(\mat X),
$$
the attributes of $\PC_q(\mat X)$ are not correlated.

$$ \cov(\mat X, \PC_q(\mat X)) = \frac 1 n \mat X' \PC_q(\mat X) = \frac 1 n \mat X' \mat X \mat B_q = \mat B_q \ve \lambda_q^d(\mat X). $$

$$ \PC_q(\mat X) \sqrt{\ve \lambda_q^{-d}(\mat X)} $$
is the normalized first~$q$ principal components, cf.~(\ref{rectangular_eigenvectors}),
because
$$ \VC \left(\PC_q(\mat X) \sqrt{\ve \lambda_q^{-d}(\mat X)} \right) = \mat I^q_p. $$

$$ \cov \left(\mat X, \ \PC_q(\mat X) \sqrt{\ve \lambda_q^{-d}(\mat X)} \right) = \mat B_q \sqrt{\ve \lambda_q^d(\mat X)}, $$
which is the correlation matrix between~$\mat X$ and the normalized first~$q$ principal components if~$\mat X$ is normalized.

If $\VC(\mat X)$ is full rank, then
\begin{multline*}
$$ \PC_p(\mat X) \sqrt{\ve \lambda_p^{-d}(\mat X)} \ \left(\PC_p(\mat X) \sqrt{\ve \lambda_p^{-d}(\mat X)} \right)'
     = \PC_p(\mat X) \left(\ve \lambda_p^d(\mat X) \right)^{-1} \PC'_p(\mat X) \\
     = \mat X \mat B_p \left(\ve \lambda_p^d(\mat X) \right)^{-1} \mat B'_p \mat X
     = \mat X \ \VC^{-1}(\mat X) \ \mat X,
\end{multline*}
which is~(\ref{mahalanobis}).


Since $\sum_i b^2_{ij} = 1$, $b^2_{ij}$ is the {\em contribution} of attribute~$i$ to principal component~$j$.

An attribute~$\ve x_{\cdot j}$ is {\em noise} iff its covariation with any other attribute is~0.
Let~$\mat X$ be normalized,
then
$ \ve u_j \in \mathfrak B (\VC(\mat X)) $
with eigenvalue~1.
If $\ve \lambda_q(\mat X) > \ve 1$ then no attribute of $\PC_q(\mat X)$ correlates with~$\ve x_{\cdot j}$,
in other words, $\PC_q(\mat X)$ represents~$\mat X$ without noise attributes.

\begin{equation*}  \label{dataReduction}
\mat X = \PC_q(\mat X) \mat B_q' + \mat \varepsilon,
\end{equation*}
where
$$ \mat \varepsilon \eqdef \PC_{-q}(\mat X) \mat B_{-q}' $$
is the {\em residual matrix}.

\proof{
$$ \PC_p(\mat X) = \mat X \mat B_p = \mat X ([\mat B_q \ \mat 0] + [\mat 0 \ \mat B_{-q}]), $$
$$ \mat X = \PC_p(\mat X) \mat B_p'
  = ([\PC_q(\mat X) \ \mat 0] + [\mat 0 \ \PC_{-q}(\mat X)])
      \times \left( \left[ \begin{array}{c} \mat B_q' \\ \mat 0 \end{array} \right] + \left[ \begin{array}{c} \mat 0 \\ \mat B_{-q}' \end{array} \right] \right)
  = \PC_q(\mat X) \mat B_q' + \PC_{-q}(\mat X) \mat B_{-q}'.
$$
}

$$ \trc \VC (\mat \varepsilon) = \trc \mat B_{-q} \ \VC(\PC_{-q}(\mat X)) \ \mat B_{-q}' = \trc \VC(\PC_{-q}(\mat X)) = \ve 1' \ve \lambda_{-q}(\mat X). $$

$$ \epsilon^2(\mat X, \mat B_q) = 2n^2 \ \trc \VC (\mat X) - 2n^2 \ \trc \VC (\PC_q(\mat X))
  = 2n^2 \left(\ve 1' \ve \lambda - \ve 1' \ve \lambda_q(\mat X) \right)
  = 2n^2 \ve 1' \ve \lambda_{-q}(\mat X)
  = 2n^2 \ \trc \VC (\mat \varepsilon)
$$
by~(\ref{sumD2}).

The average lost squared distance
$$ \frac {\epsilon^2(\mat X, \mat B_q)} {n^2} = 2 \ \trc \VC (\mat X) (1 - r_q), $$
where the {\em explained fraction of variance}, a quality criterion,
\begin{equation}\label{explFracVar}
r_q \eqdef \frac {\ve 1' \ve \lambda_q(\mat X)} {\trc \VC (\mat X)}
  = \frac {\trc \VC (\PC_q(\mat X))} {\trc \VC (\mat X)}
  = \frac {\sum_{i,j} y^2_{ij}} {\sum_{i,j} x^2_{ij}}
  = \frac {\sum_{i,j} d^2(\ve y_i, \ve y_j)} {\sum_{i,j} d^2(\ve x_{i\cdot}, \ve x_{j\cdot})},
\end{equation}
where $\mat Y \eqdef \PC_q(\mat X)$.
$$ 0 \le r_q \le 1. $$

\comm {
The data matrix~$\mat Y$ can be corrected $\PC_q(\mat X)$ so that
$$ \sum_{i,j} d^2(\ve x_{i\cdot}, \ve x_{j\cdot}) = \sum_{i,j} d^2(\ve y_i, \ve y_j) $$
by these methods:
\begin{enumerate}
\item \begin{equation}\label{PCDataInflated} \mat Y \eqdef (1 / \sqrt {r_q}) \times \PC_q(\mat X). \end{equation}
\item $\mat Y \eqdef [\PC_q(\mat X) \ \delta \mat I_n]$, where
$$ \delta \eqdef \sqrt{\frac {\epsilon^2(\mat X, \mat B_q)} {2n(n-1)}}. $$
$$ \delta = \sqrt {\frac n  {n-1} \left(\trc \VC(\mat X) - \ve 1' \ve \lambda_q(\mat X) \right)}. $$
\proof{
$$ \sum_{i,j} d^2(\ve y_i, \ve y_j) = \sum_{i,j} d^2(\ve x_{i\cdot}' \mat B_q, \ve x_{j\cdot}' \mat B_q) + n(n-1) \times 2 \delta^2
  = \sum_{i,j} d^2(\ve x_{i\cdot}' \mat B_q, \ve x_{j\cdot}' \mat B_q) + \epsilon^2(\mat X, \mat B_q) = \sum_{i,j} d^2(\ve x_{i\cdot}, \ve x_{j\cdot}).
$$
}
\end{enumerate}
}

The operation $\PC_q(\mat X)$ is idempotent:
$$ \PC_q(\PC_q(\mat X)) = \PC_q (\mat X) \ \mat I_q \ \mat I^q_q = \PC_q(\mat X), $$
$$ \ve \lambda_q (\PC_q(\mat X)) = (\mat I^q_q)' \ \ve \lambda(\VC(\PC(\mat X))) = \ve \lambda_q(\mat X), $$
because
$$ \mathfrak B(\VC(\PC_q(\mat X))) = \mathfrak B(\ve \lambda_q^d(\mat X)) = \{\mat I_q \}, $$
$$ \ve \lambda(\VC(\PC_q(\mat X))) =  \ve \lambda(\ve \lambda_q^d(\mat X)) = \ve \lambda_q(\mat X). $$



\subsection{Affine component analysis}

Let~$\mat X \in \R^{n \times p}$ be a data matrix,
s.t.~$\E \mat X = \ve 0$.

The attributes of~$\mat X$ are merged by an affine combination~$\ve \beta$ with a minimum variance iff
$$ \ve \beta = \frac {\VC^{-1}(\mat X) \ve 1} {\ve 1' \VC^{-1}(\mat X) \ve 1}, $$
cf.~(\ref{betaL2_stat}).
$$ \ve 1' \ve \beta = 1. $$
\proof
{
$$ 0 = \frac d {d \ve \beta} \left[ \var \mat X \ve \beta - \lambda (\ve 1' \beta - 1) \right]
     = \frac d {d \ve \beta} \left[ \ve \beta' \VC(\mat X) \ve \beta - \lambda (\ve 1' \beta - 1) \right]
     = 2 \VC(\mat X) \ve \beta - \lambda \ve 1, $$
where $\lambda$ is a Lagrange multiplier.
}

$$ \var \mat X \ve \beta = \ve \beta' \VC(\mat X) \ve \beta = \frac 1 {\ve 1' \VC^{-1}(\mat X) \ve 1}. $$
\comm {$< 0$}
\comm
{Select a subset of attributes $\mat S \subseteq \mat X$, s.t.~$\var \mat S \ve \beta$ is minimum,
    i.e.~$\ve 1' \VC^{-1}(\mat S) \ve 1$ is maximum.
}

$$ \forall \mu \in \R, \ve \beta \in \R^p \
  (\E \mat Y = \mu \ve 1, \ \ve y \eqdef \mat Y \ve \beta, \ \mat X \eqdef \mat Y - \ve y \ve 1' \implies \E {\mat X} = \ve 0).
$$
\proof
{
    $$ \ve 1' \mat X = \ve 1' (\mat Y - \ve y \ve 1')
    = \ve 1' \mat Y - \ve 1' \mat Y \ve \beta \ve 1'
    = \ve 1' \mat Y \ (\mat I - \ve \beta \ve 1')
    = n \mu \ve 1' \ (\mat I - \ve \beta \ve 1')
    = n \mu \ (\ve 1' - \ve 1')
    = \ve 0.
    $$
}

\comm {
Let~$\mat Y \in \R^{n \times p}$ be a data matrix,
s.t.
$$ \mat Y = \ve y \ve 1' + \mat X, $$
then the new attribute
$$ \mat Y \ve \beta = \ve y + \mat X \ve \beta $$
is the {\em combining attribute of~$\mat Y$}.
$$ \E \mat Y \ve \beta = \ve y. $$
$$ \var \mat Y \ve \beta = \var \mat X \ve \beta. $$
}

\comm{Other affine components}



\subsection {Factor analysis}
\comm{}


\subsection {Classical multidimensional scaling}
\comm {of a distances or similarities}

Let~$\mat S \in \R^{n \times n}$ be a double-centered symmetric matrix.
$$ rank(\mat S) \le n - 1. $$

{\em Classical multidimensional scaling} (cMDS) is a procedure of the construction of a data matrix~$\mat Y \in \R^{n \times q}$, where $q \le n$,
such that $\mat S(\mat Y)$ approximates~$\mat S$.

Let
$$ \mat B \in \mathfrak B (\mat S), $$
$$ \ve \lambda \eqdef \lambda(\mat S), $$
and let data matrix $\mat X \in \C^{n \times n}$ be defined as
$$ \mat X \eqdef \mat B \sqrt{\ve \lambda^d}, $$
then
$$ \mat S = \mat S(\mat X). $$

By~(\ref{sim2dataCentered}), $\bar {\mat X} = \ve 0$.

$$ \mathfrak B (\mat X' \mat X) = \{\mat I\}. $$

$$ \mat X_q \eqdef \mat X \mat I^q_n. $$

The {\em multidimensional scaling} of~$\mat S$ with $q$ attributes is
$$ \MDS_q(\mat S) \eqdef \mat X_q. $$
$$ \MDS_q(\mat S) \in \R^{n \times q}. $$

By the Eckart-Young theorem for squared matrices,
$$ \mat X_q \in \arg \min_{\mat Y \in \R^{n \times q}} d^2(\mat S, \mat Y \mat Y'), $$
\comm {relation to $d^2(\mat D^2, \mat D_{\mat S}(\mat Y \mat Y'))$}
$$ \min_{\mat Y \in \R^{n \times q}} d^2(\mat S, \mat Y \mat Y') = \sum_{i = q+1}^n \lambda_i^2. $$

If $i \le q$ and $\lambda_i < 0$ then the column~$\ve x_{\cdot i}$ of~$\mat X_q$ is imaginary and the procedure {\em fails}.

Suppose $\forall i \le q : \lambda_i > 0$.
\comm{negative $\lambda_i$}

By the Eckart-Young theorem for rectangular matrices,
$$ \mat X_q \in \arg \min_{\mat Y \in \R^{n \times q}} d^2(\mat X, \mat Y), $$
$$ \min_{\mat Y \in \R^{n \times q}} d^2(\mat X, \mat Y) = \sum_{i = q+1}^n \lambda_i. $$

Since
$$ \VC (\mat X_q)  = \frac 1 n \mat X_q' \mat X_q = \frac 1 n \mat I^q_n \ve \lambda^d \mat I^q_n = \frac 1 n \ve \lambda_q^d, $$
the explained fraction of variance is
$$ r_q = \frac {\trc \VC (\mat X_q)} {\trc \VC (\mat X_n)} = \frac {\trc \ve \lambda^d_q} {\trc \mat S} $$
by~(\ref{explFracVar}) and~(\ref{trc_S_VC}).

$$ r_q = 1 - \min_{\mat Y \in \R^{n \times q}} d^2(\mat X, \mat Y) / \trc \mat S. $$
It can be that $r_q \not \in [0,1]$.

$$ \trc \MDS_q(\mat S) \ \MDS_q'(\mat S) = \trc \mat X_q \mat X_q' = \trc \ve \lambda^d_q = r_q \ \trc \mat S. $$

Converting a similarity vector~$\ve s$ into an object~$\ve x$:
$$ \ve s = \mat X \ve x \implies \ve x = \sqrt{\ve \lambda^{-d}} \mat B' \ve s. $$
\proof
{
$\mat B' \ve s = \mat B' \mat B \sqrt{\ve \lambda^d} \ve x = \sqrt{\ve \lambda^d} \ve x.$
}

\comm { distance $\to$ distance$^p$, criterion for the best p}

Since
$$ \trc \mat S^2 = \trc \left(\ve \lambda^d \right)^2, $$
the {\em explained fraction of squared similarity}
$$ s_q \eqdef \frac {\trc \left(\ve \lambda^d_q \right)^2} {\trc \mat S^2} $$
is between~0 and~1.

$$ s_q = 1 - \min_{\mat Y \in \R^{n \times q}} d^2(\mat S, \mat Y \mat Y') / \trc \mat S^2. $$

\comm
{model: $\mat S = \mat Y \mat Y' + \mat {Noise}$, where $\mat{Noise} = \mat N \mat N'$, s.t.~$\mat N \in \C^{n \times p}$ and $\VC(\mat N) = 1 / (\trc \mat S) \ \mat I$.

The distribution of $\{\lambda_i : i\}$ should be a mixture of Zipf distributions with consecutive~$i$s: practical examples.
The component with largest $\lambda$s is not noise.
Hence the $q$ selection.

If distance is geographical then there should be no noise.
}


\subsubsection{Relation to principal component analysis}

If a data matrix $\mat Y \in \R^{n \times p}$, $\mat M \eqdef \MDS_q(\mat S(\mat Y))$ and $\mat P \eqdef \PC_q(\mat Y)$,
then their~$r_q$ is the same,
$$ \lambda_M = n \ \lambda_P, $$
$$ \mat Y' \mat B_M \sqrt{\lambda_M^{-d}} \in \mathfrak B_P, $$
$$ \mat Y  \mat B_P \sqrt{\lambda_P^{-d}} \in \mathfrak B_M. $$
If
\begin{equation} \label{M2P_match}
  \mat B_P = \mat Y' \mat B_M \sqrt{\lambda_M^{-d}},
\end{equation}
\begin{equation} \label{P2M_match}
  \mat B_M = \mat Y  \mat B_P \sqrt{\lambda_P^{-d}}
\end{equation}
then
$$ \mat M \mat M' = \mat B_M \sqrt {\lambda_M^d} \sqrt {\lambda_M^d} \mat B_M'  = \mat Y \mat B_P \mat B_P' \mat Y' = \mat P \mat P', $$
which means that the distances between the objects of~$\mat P$ and~$\mat M$ are the same.

If all elements of~$\lambda_P$ are different, then~(\ref{M2P_match}) and~({\ref{P2M_match}}), and $\mat P = \mat M$.
Therefore, the attributes constructed by MDS are also referred to as principal components.



\subsubsection{Multidimensional scaling of attributes}

If~$\mat Y \in \R^{m \times n}$ is a data matrix, $\bar {\mat Y} = \ve 0$ and $\mat S = \VC (\mat Y)$
then the objects of $\MDS(\mat S)$ are the attributes of~$\mat Y$.
$$ \mat S \succeq 0. $$
$$ \mat X_n = \cov \left(\mat Y, \ \PC_m(\mat Y) \sqrt{\ve \lambda^{-d}} \right), $$
where $\ve \lambda \eqdef \ve \lambda(\VC(\mat Y))$.

If the attributes of~$\mat Y$ are normalized
then for $\mat Z \eqdef (1/\sqrt 2) \mat X_q$
$$ \cor(\ve y^i, \ve y^j) \le 1 - d^2(\ve z_i, \ve z_j) $$
by~(\ref{d2rho}),
and this inequality is tight if $q = n$.


\subsection {Canonical correlations}
\comm{}



\subsection{Clustering}

Let a {\em clustered data} be
$$ (\mat Y, \ve w_T) \eqdef
\left (
\left[
\begin{array}{c}
\mat Y_1 \\
\mat Y_2 \\
\dots \\
\mat Y_q
\end{array}
\right],
\left[
\begin{array}{c}
\ve w_1 \\
\ve w_2 \\
\dots \\
\ve w_q
\end{array}
\right]
\right)
$$
where $\mat Y_k \in \R^{n \times p}$ and $q$ is the number of clusters.

The part $(\mat Y_k,\ve w_k)$ of $(\mat Y, \ve w_T)$ will be referred to as {\em cluster~$k$}.

The {\em probability of cluster~$k$}
$$ p_k \eqdef \frac {\ve 1' \ve w_k} {\ve 1' \ve w_T}. $$

The {\em probability of object~$i$ to be in cluster~$k$}
$$ p_{ik} \eqdef \frac {w_k[i]} {\sum_{l=1}^q w_l[i]}. $$

Then
\begin{equation*}
  \VC(\mat Y) = \sum_{k=1}^q p_k \VC(\mat Y_k).
\end{equation*}

\bigskip
Let $(\mat X, \ve w)$, where $\mat X \in \R^{n \times p}$, be a data, then a {\em clustering} of $(\mat X, \ve w)$ is a clustered data
$$ (\mat T, \ve w_T) \eqdef
   \left (
     \left[
       \begin{array}{c}
         \mat X_1 \\
         \mat X_2 \\
         \dots \\
         \mat X_q
       \end{array}
     \right],
     \left[
       \begin{array}{c}
          \ve w_1 \\
          \ve w_2 \\
          \dots \\
          \ve w_q
       \end{array}
     \right]
   \right)
$$
where
$$ \mat X_1 = \mat X_2 = \dots = \mat X_q = \mat X $$
and
$$ \sum_{i=1}^q \ve w_k = \ve w. $$
The data matrix~$\mat T$ is the {\em total clustering} data matrix.


\subsubsection{Ellipsoid clustering}

\bigskip
$$ \VC(\mat X) = \VC (\mat T). $$

\bigskip
Let~$\mat B_k \in \R^{n \times p}$ be a matrix where each row is~$\bar {\mat X}_k$,
then the {\em between-clusters} clustered data matrix
$$ \mat B(\mat T) \eqdef
     \left[
       \begin{array}{c}
         \mat B_1 \\
         \mat B_2 \\
         \dots \\
         \mat B_q
       \end{array}
     \right].
$$
$$ rank(\mat B(\mat T)) \le q. $$
$$ \bar {\mat X} = \ve 0 \implies rank(\mat B(\mat T)) \le q - 1. $$
The {\em within-clusters} clustered data matrix
$$ \mat W(\mat T) \eqdef \mat T - \mat B(\mat T). $$
$$ \mat T = \mat B(\mat T) + \mat W(\mat T). $$

\bigskip
$$ \forall \mat A : \mat B(\mat T \mat A) = \mat B(\mat T) \mat A. $$
$$ \forall \mat A : \mat W(\mat T \mat A) = \mat W(\mat T) \mat A, $$
because
$$ \mat W(\mat T \mat A) = \mat T \mat A - \mat B(\mat T \mat A) = \mat T \mat A - \mat B(\mat T) \mat A = (\mat T - \mat B(\mat T)) \mat A. $$

\bigskip
$$ \VC(\mat W_k) = \frac 1 n (\mat X_k - \mat B_k)' (\mat X_k - \mat B_k) = \frac 1 n \mat X_k \mat X_k - \bar {\mat X}_k \bar {\mat X}_k' = \VC(\mat X_k) - \VC(\mat B_k). $$
$$ \VC(\mat X) = \VC(\mat T) = \sum_{k=1}^q p_k \VC(\mat X_k) = \sum_{k=1}^q p_k \VC(\mat B_k) + \sum_{k=1}^q p_k \VC(\mat W_k) = \VC(\mat B(\mat T)) + \VC(\mat W(\mat T)). $$
$$ \trc \VC(\mat X) = \trc \VC(\mat B(\mat T)) + \trc \VC(\mat W(\mat T)). $$
A quality criterion, the explained fraction of variance of the clustering $(\mat T, \ve w_T)$ is
$$ r_q = \frac {\trc \VC(\mat B(\mat T))} {\trc \VC(\mat X)} = \frac {\ve 1' \mat D^2_{\mat B} \ve 1} {\ve 1' \mat D^2 \ve 1}
$$
by~(\ref{sumD2}),
which defines a {\em clustering for a distance matrix}.
\comm {compute $\mat D^2_{\mat B}$}
\comm {This is the K-means criterion}
\comm {Multi-normal mixture criterion}


\subsubsection {Distinct clusters}
The clusters of a clustering are {\em distinct} iff there is a function $c : [n] \mapsto [q]$, s.t.
\begin{equation*}
  w_k[i] =
    \begin{cases}
      w[i], &\text{ if } c(i) = k\\
      0,   &\text{ else}.
    \end{cases}
\end{equation*}

Cluster~$k$
$$ C_k \eqdef \{i \in [n] \mid w_k[i] > 0\}. $$

The {\em size} of cluster~$k$
$$ n_k \eqdef |C_k|. $$

For ellipsoid clustering
$$ rank(\mat W_k) \le n_k - 1. $$
$$ rank(\mat W(\mat T)) \le \sum_k rank(\mat W_k) \le n - q. $$


\subsubsection{Canonical component analysis}
By the canonical analysis (Section \ref{CanonicalAnalysis}),
the local extrema of $\{\ve x' \VC(\mat B(\mat T)) \ve x \mid \ve x' \VC (\mat W(\mat T)) \ve x = 1\}$
are the linear combinations of the attributes of~$\mat X$.
The constructed attributes are referred to as {\em canonical components}
\comm {extrema = maxima}
$$ \CC_q \eqdef \mat X \mat R, $$
where
$$ \mat R \eqdef \mat U^{-1} \mat C, $$
$$ \mat C \in \mathfrak B \left((\mat U')^{-1} \VC(\mat B(\mat T)) \mat U^{-1} \right) $$
and $\VC(\mat W(\mat T)) = \mat U' \mat U$ is a Cholesky decomposition.

$$ \ve \lambda \eqdef \ve \lambda \left((\mat U')^{-1} \VC(\mat B(\mat T)) \mat U^{-1} \right). $$

$$ rank (\ve \lambda^d) \le rank(\mat B(\mat T)). $$
If the clusters are distinct then
$$ rank (\ve \lambda^d) \le rank(\mat W(\mat T)). $$

$$ \mat R' \ \VC(\mat X) \ \mat R = \mat R' \ \VC(\mat B(\mat T)) \ \mat R + \mat R' \ \VC(\mat W(\mat T)) \ \mat R. $$
$$ \VC (\CC_q) = \ve \lambda^d + \mat I $$
by~(\ref{canonicalBetween}) and~(\ref{canonicalWithin}).

A quality criterion is $\ve 1' \ve \lambda$.

\bigskip
$$ \mat Y \eqdef \mat T \mat R $$
is a clustered data.
$$ \mat R' \ \VC(\mat B(\mat T)) \ \mat R = \VC(\mat B(\mat Y)). $$
$$ \mat R' \ \VC(\mat W(\mat T)) \ \mat R = \VC(\mat W(\mat Y)). $$
$$ \VC(\mat Y) = \VC(\mat B(\mat Y)) + \VC(\mat W(\mat Y)). $$
\comm{relation to discriminant analysis}

\bigskip
If $\bar {\mat X} = \ve 0$, $q = 2$ and $\mat X \ne \mat 0$
then $rank (\mat B(\mat T)) = 1$
and, removing zero eigenvalues,
$$\VC(\mat B(\mat Y)) = \var \ve b(\ve y) = \lambda, $$
$$ \var \ve b(\ve y) = - \ {\bar {\ve y}}_1 \ {\bar {\ve y}}_2. $$
$$ \bar {\ve y} = 0. $$
$$ \var \ve y = \lambda + 1. $$



\section{Prediction}
Let~$\mat X \in \R^{n \times p}$ be a data matrix and $\ve y \in \R^n$ be a {\em target} attribute.
Attribute~$\hat {\ve y}=f(\mat X)$ is a {\em prediction} of~$\ve y$ given data matrix~$\mat X$.
The function~$f$ will be referred to as the {\em predictor}.
\comm { Given $F$, an {\em $F$-predictor} is the set of parameters defining $f \in F$. {\em parametric predictor}.}

The {\em residual} attribute
$$ \ve \epsilon \eqdef \ve y - \hat {\ve y}. $$

The {\em absolute criterion} in metric $L_k$ is
$$ \sum_{i=1}^n |\epsilon_i|^k. $$

The {\em prediction error} is the distance between $\ve y$ and $\hat {\ve y}$ in metric $L_k$ normalized by the number of objects
$$ E \eqdef \left(\frac 1 n \sum_{i=1}^n |\epsilon_i|^k \right)^{1/k}. $$
The prediction error is a quality criterion.
$$ E \ge 0. $$

A {\em best predictor}
\begin{equation} \label{bestPredictor}
f^* \eqdef \arg \min_{f \in F} E,
\end{equation}
where $F$ is a set of predictors.

The metric $L_k$ will be referred to as the {\em prediction metric}.


\subsection{Statistical models for prediction in metric $L_2$}
If~$y_i$ are random variables and~$\hat y_i$ are constants then
$$ \var \epsilon_i = \var y_i. $$
If $\E \epsilon_i = 0$ then
$$ \var \epsilon_i = \E \epsilon^2_i - (\E \epsilon_i)^2 = \E \epsilon^2_i. $$
If
\begin{equation} \label{lrContrib}
w_i \eqdef \frac 1 {\var y_i}
\end{equation}
then all objects have the same contribution to~$E$.

Alternatively,
if $\epsilon_i \sim N(0,\var \epsilon_i)$ and all $y_i$ are independent of each other
then the negative logarithm of the likelihood of~$\ve y$ is
$$ \sum_i \frac {\epsilon^2_i} {\var \epsilon_i} + const $$
and the maximum likelihood estimate of the predictor is the best predictor~(\ref{bestPredictor}) with multiplicities~(\ref{lrContrib}).

Since
$$ \sqrt{w_i} \ \epsilon_i = \frac {\epsilon_i} {\sqrt {\var \epsilon_i}}  \sim N(0,1), $$
which can be used to find outlier~$y_i$.

\comm { consistent estimate of the regression function}



\subsection {Linear prediction} \label{linearPrediction}
A {\em linear} predictor
$$ f(\mat X) \eqdef \mat X \ve \beta, $$
where {\em predictor vector} $\ve \beta \in \R^p$.
\comm { Prove that there is one local optimum ?}


\subsubsection {Linear prediction in metric $L_2$} \label{linearPredictionL2}
$$ E = \frac {|\ve \epsilon|} {\sqrt{n}}. $$
The prediction error~$E$ is minimum if $|\ve \epsilon|$, i.e., the Euclidean distance between $\ve y$ and $\hat {\ve y}$, is minimum.
Therefore, $\hat {\ve y}$ is the projection of~$\ve y$ on the space spanned by the attributes of~$\mat X$.

By Pythagorean theorem,
$$ |\ve y|^2 = |\hat {\ve y}|^2 + |\ve \epsilon|^2. $$
$$ \hat {\ve y}' \ve \epsilon = 0. $$
If there is a dependence between~$\hat {\ve y}$ and~$\ve \epsilon$ then prediction can be improved.
The dependence can be found by plottig objects in the space $\{\hat {\ve y}, \ve \epsilon\}$.
$$ \hat {\ve y}' \ve y = |\hat {\ve y}|^2. $$
$$ |\ve \epsilon|^2 = \ve y' \ve \epsilon. $$
The prediction error is the {\em root mean squared error, RMSE}.

The {\em explained fraction of variance}
$$ r \eqdef \frac {|\hat {\ve y}|^2} {|\ve y|^2} = 1 - \frac {|\ve \epsilon|^2} {|\ve y|^2} = 1 - \frac {E^2} {|\ve y|^2 / n}, $$
which is between~0 and~1, cf.~(\ref{explFracVar}).
The explained fraction of variance is a quality criterion.

The best linear predictor
\begin{equation} \label{betaOpt}
  \ve \beta^* = \arg \min_{\ve \beta \in \R^p} |\ve \epsilon|^2.
\end{equation}

In the sequel $\ve \beta \eqdef \ve \beta^*$.
$$ \frac d {d \ve \beta} |\ve \epsilon|^2 = \frac d {d \ve \beta} |\ve y - \mat X \ve \beta|^2 = - 2 \mat X' (\ve y - \mat X \ve \beta) = 0. $$
$$ \mat X' \ve y = \mat X' \mat X \ve \beta = \mat X' \hat {\ve y}. $$
$$ \mat X' \ve \epsilon = \mat X' \ve y - \mat X' \hat {\ve y} = \ve 0. $$
$$ \exists i : \ve x_{. i} = \ve 1 \times const \implies \bar{\ve y} = \bar{\hat {\ve y}} \text{ and } \bar{\ve \epsilon} = 0, $$
because $\ve 1' \ve y = \ve 1' \hat {\ve y}$.

Assuming $(\mat X' \mat X)^{-1}$ exists,
\begin{equation} \label{betaL2}
  \ve \beta = (\mat X' \mat X)^{-1} \mat X' \ve y
\end{equation}
with the running time $O(p^2 n)$.

\begin{equation} \label{betaL2_stat}
   \ve \beta = \VC^{-1} (\mat X) \ \cov(\mat X,[\ve y]).
\end{equation}
$$ \ve \beta = \mat X^- \ve y. $$
\comm{Which of generalized inverses}

$$ \hat {\ve y} = \mat X (\mat X' \mat X)^{-1} \mat X' \ve y. $$
$$ |\hat {\ve y}|^2 = \ve \beta' \mat X' \ve y. $$

The {\em coefficient of determination}
$$ R^2 \eqdef 1 - \frac {|\ve \epsilon|^2 / n} {\var \ve y}. $$
$$ \bar {\ve y} = 0 \implies r = R^2 = \cor^2 (\ve y, \hat {\ve y}). $$

If $\bar {\mat X} = \ve 0$, $\bar {\ve y} = 0$ and $\forall i, j: i \ne j \implies \cov(\ve x_i, \ve x_j) = 0$ then
\begin{equation} \label{betaIndep}
\beta_i = \frac {\cov (\ve x_i, \ve y)} {\var \ve x_i} = \rho(\ve x_i, \ve y) \sqrt{\frac {\var \ve y} {\var \ve x_i}}
\end{equation}
and
$$ \var \hat {\ve y} = \sum_i \beta_i^2 \ \var \ve x_i = \sum_{i : \var \ve x_i \ne 0} \frac {\cov^2 (\ve x_i, \ve y)} {\var \ve x_i}. $$
Let
$$ a_i \eqdef \frac {\cov^2 (\ve x_i, \ve y)} {\var \ve x_i} $$
be the {\em contribution} of~$\ve x_i$ to the prediction.
\comm { Relation of contribution to the quality criterion ?}

If $\mat X = [\ve 1 \ \ve x]$ then
$$ \beta_2 = \frac {\cov (\ve x, \ve y)} {\var (\ve x)}, $$
$$ \beta_1 = \bar {\ve y} - \beta_2 \bar {\ve x}. $$


\subsubsection {Finding a linear predictor by alternating optimization}

Let $A \eqdef \{i,j,\dots,q\} \subseteq [p]$,
and $L_k$ be a prediction metric.

If
$$ \mat X_A \eqdef [\ve x^i, \ve x^j, ..., \ve x^q], $$
$$ \ve \beta_A \eqdef [\ve \beta_i, \ve \beta_j, ..., \ve \beta_q], $$
$$ \hat {\ve y}_A \eqdef \mat X_A \ve \beta_A, $$
$$ \ve y_A \eqdef \ve y - (\hat {\ve y} - \hat {\ve y}_A), $$
$$ \ve \epsilon_A \eqdef \ve y_A - \hat {\ve y}_A $$
then
$$ \ve \epsilon_A = \ve \epsilon $$
and the best linear predictor~$\ve \beta^*_A$ of~$\ve y_A$ by~$\mat X_A$ optimizes~$\ve \beta$.

Alternating optimization approximation algorithm to find~$\ve \beta^*_A$:
\begin{enumerate}
  \item $\ve \beta := $ arbitrary values;
  \item while~$|\ve \epsilon^2|$ decreases significantly:
    \label{alg:altOptLinPredLoop}
    \begin{enumerate}
        \item $A := \textrm {arbitrary subset of } [p]$;
        \item $\ve \beta^*_A :=$ best linear predictor of~$\ve y_A$ by~$\mat X_A$;
        \item $[\ve \beta_i, \ve \beta_j, ..., \ve \beta_q] := [\ve \beta^*_i, \ve \beta^*_j, ..., \ve \beta^*_q]$.
    \end{enumerate}
\end{enumerate}

If $k = 2$ and the loop of step~\ref{alg:altOptLinPredLoop} is repeated $p/a$ times, then the running time is $p/a \times O(a^2 n) = O(pan)$,
where $a \eqdef |A|$.

\comm { Prove that a global optimum is reached }

\comm { Global optimum can be reached even if linear constraints on $\ve \beta$ are applied }
\vspace {5mm}

\comm { One attribute: $|\ve \epsilon|^2$ is the same whether y is predicted based on x, or x is predicted based on y}

\comm{partial least squares}






\end{document}


