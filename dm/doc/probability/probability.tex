\documentclass[10pt,a4paper]{article}
\usepackage[latin1]{inputenc}
\usepackage{amsmath}
\usepackage{amsfonts}
\usepackage{amssymb}
\usepackage[]{theorem}



\setlength{\topmargin}{-.5in}
\setlength{\textheight}{9in}
\setlength{\oddsidemargin}{.125in}
\setlength{\textwidth}{6.25in}

\theoremstyle{plain} \newtheorem{Lem}{Lemma}


\newcommand {\subsubsubsection} [1] {\vskip 0.5em {\bf \em #1}}

%\newcommand {\comm} [1] {}
\newcommand {\comm} [1]
{
\par
\setlength{\leftskip}{-0.8in}
\begin{tabular} {p{0.1cm} | p{\textwidth}}
  {\bf ?} & {#1}
\end{tabular}
\par
\setlength{\leftskip}{0in}
}


\newcommand {\cond} {, & \textrm{if }}
\newcommand {\otherwise} {, & \textrm{else}}

\newcommand {\Forall} {\forall \ }
\newcommand {\eqdef} {\triangleq}
\newcommand {\equivalent} {\Leftrightarrow}
\renewcommand {\implies} {\Rightarrow}
\renewcommand {\impliedby} {\Leftarrow}
\renewcommand {\And}  {\ \& \ }
\newcommand {\Or}  {\ \vee \ }
\newcommand {\QED} {$\square$ \ \par }
\newcommand {\proofword}  {{\par \sf Proof:\ }}
\newcommand {\proof} [1] {\proofword #1 \QED \par}
\newcommand {\definition} [2] {\bigskip {\bf Definition} {\rm \sc #1} \par #2 \QED \bigskip}
\newcommand {\example} [2] {\bigskip {\bf Example.} {#1} \par #2 \QED}
\newcommand {\theorem} [2] {\bigskip {\bf Theorem} {\rm \sc #1} \par #2 \QED \bigskip}
\newcommand {\ve} {\mathbf}
\newcommand {\vecc} {\boldsymbol}
\newcommand {\Questions} {\vspace{5mm} {\bf Unresolved Questions:}}
\newcommand {\assert} {\square \ }

\renewcommand {\P} {\mathrm P}
\newcommand {\E} {\mathrm E \ }
\newcommand {\var} {\mathrm {var} \ }
\newcommand {\cov} {\mathrm {cov} }
\newcommand {\cor} {\mathrm {cor} }
\newcommand {\N} {\mathbb N}
\newcommand {\Z} {\mathrm Z}
\newcommand {\R} {\mathbb R}
\newcommand {\C} {\mathbb C}
\newcommand {\I} {\mathbb I}
\newcommand {\supp} {\mathrm {Supp}}
\newcommand {\mat} {\mathrm}
\newcommand {\trc} {\mathrm {tr} \ }
\newcommand {\VC} {\mathrm {VC} }
\newcommand {\const} {\mathrm {const} }
\newcommand {\sgn} {\mathrm {sgn} }
\newcommand {\dom} {\mathrm {dom} }
\newcommand {\range} {\mathrm {range} }
\newcommand {\Arg} {\mathrm {Arg} \ }

\newcommand {\LCA} {\mathrm {LCA} }  % least common ancestor (in a tree)

\newcommand {\nul} {\mathrm {null}}
\newcommand {\Dt} {\Delta t}
\newcommand {\PC} {\mathrm {PC} }
\newcommand {\MDS} {\mathrm {MDS} }
\newcommand {\CC} {\mathrm {CC} }

\newcommand {\re} {\mathrm {re}}
\newcommand {\cl} {\mathrm {cl}}
\newcommand {\intr} {\mathrm {int}}

\newcommand {\<} {\langle}
\renewcommand {\>} {\rangle}

\newcommand {\dotve} [1] {\dot {\ve #1}}

\newcommand {\la} [1] {\lambda (#1) \ }

\newcommand {\lhs} {\mathrm {lhs}}
\newcommand {\rhs} {\mathrm {rhs}}

\providecommand{\keywords}[1]
{
  \small
  \textbf{\textit{Keywords---}} #1
}

\makeatletter
\DeclareRobustCommand{\cev}[1]{%
  {\mathpalette\do@cev{#1}}%
}
\newcommand{\do@cev}[2]{%
  \vbox{\offinterlineskip
    \sbox\z@{$\m@th#1 x$}%
    \ialign{##\cr
      \hidewidth\reflectbox{$\m@th#1\vec{}\mkern4mu$}\hidewidth\cr
      \noalign{\kern-\ht\z@}
      $\m@th#1#2$\cr
    }%
  }%
}
\makeatother




\title{Probability}
\author{Vyacheslav Brover}


\begin{document}
\maketitle
\tableofcontents


\section{Notation}

$$ Prob \eqdef [0,1]. $$

\begin{tabular}{|c|c|c|}
\hline & Random & Sample \\
\hline 
\hline Variable & $X$ & $x$ \\
\hline Vector   & $\ve X$ & $\ve x$ \\
\hline Matrix   & $\mathcal X$ & $\mat X$ \\
\hline
\end{tabular}


\section{Discrete random variable}
Let~$X$ be a discrete random variable, $\supp(X) = [a,b]\cap \N$ and $p_i \eqdef \P(X=i)$.

Let 
$$\bar F_i \eqdef 1 - \P(X \le i) = \sum_{j=i+1}^b p_j$$
be the {\em reverse probability function} of~$X$.

Then
\begin{equation}
 \sum_{i=a}^b \bar F_i = \sum_{i=a}^b p_i (i-a) = \E X - a (b-a+1).  \label{sum_rev_pf}
\end{equation}



\section {Power law distribution}

Let~$X \in \R$ be a random variable, s.t.~$0 \le x_{\mathrm {min}} \le X \le x_{\mathrm {max}} \le \infty $.

Then $X \sim Power(x_{\mathrm {min}},x_{\mathrm {max}},\gamma)$ if the p.d.f.~of~$X$ is
$$ \phi(x) \eqdef c \gamma \ x^{- (\gamma + 1)}. $$
$$ \P(X > x) = c \ \left(x^{-\gamma} - x_{\mathrm {max}}^{-\gamma} \right). $$
$$ c = \frac 1 {x_{\mathrm {min}}^{-\gamma} - x_{\mathrm {max}}^{-\gamma}}. $$

$$ x_{\mathrm {max}} = \infty \implies \gamma > 0, \ x_{\mathrm {min}} > 0. $$
$$ x_{\mathrm {min}} = 0 \implies \gamma < 0, \ x_{\mathrm {max}} < \infty. $$
It follows that
$$ \supp(X) \ne \R_+. $$

$$ \E \log X 
= c \left(  x_{\mathrm {min}}^{-\gamma} \ \log x_{\mathrm {min}} 
- x_{\mathrm {max}}^{-\gamma} \ \log x_{\mathrm {max}} 
\right)
+ \frac 1 \gamma.
$$
$$ x_{\mathrm {max}} = \infty \implies \E \log X = \log x_{\mathrm {min}} + \frac 1 \gamma. $$
$$ x_{\mathrm {min}} = 0      \implies \E \log X = \log x_{\mathrm {max}} + \frac 1 \gamma. $$

Let $x_1 \ge x_2 \ge \dots \ge x_n$ be a random sample of~$X$,
then
$$ \hat \P(X \ge x_i) = \frac i n $$
and
$$ x_i \approx  k \ i^{- \alpha}, $$
where $k = (nc)^{\frac 1 \gamma}$ and $\alpha = \frac 1 \gamma$.
This is a {\em Zipf law}.

A power law distribution can be visualized as a straight line 
if~$i$ and~$x_i$ are drawn in logarithmic scales.

\comm{Beta1 distribution}


\subsection{Principle of maximum entropy}

Let~$X \in (0,1]$ be a random variable, and $Y \eqdef - \log X$, so that $Y \in [0,\infty)$.
By the principle of maximum entropy, it may be that $Y \sim Exponential(\lambda)$, where $\lambda > 0$ and $\E Y = 1/\lambda$.
Then $X \sim Beta(\lambda,1)$ with p.d.f.
$$ \phi(x) = \lambda x^{\lambda - 1}. $$

\comm {Beta}

For example, $X$ is the probability of a message, and $Y$ is the length of the message, 
and $X$ and~$Y$ are mutually optimized so that the mean message length is minimum.


\subsection{$Beta(\alpha,1)$ distribution}

Consider a nominal variable~$X$ with an infinite number of values~$i \in \N \setminus \{0\}$.
Let~$p_i$ be the probability of value~$i$ and $p_i \ge p_{i+1}$.
Let~$p_i=f(i)$ be a continuous function.
Then $\phi(p) \eqdef p [- f^{-1}(p)]'$ is the p.d.f.~of $p_X$.
By the principle of maximum entropy, $p_X \sim Beta(\alpha,1)$:
$$ p [- f^{-1}(p)]' = \alpha p^{\alpha - 1}, $$
$$ f^{-1}(p) = \frac \alpha {1 - \alpha} \ p^{-(1-\alpha)} = i, $$
$$ f(i) = \left(\frac {1-\alpha} \alpha \ i \right)^{-\frac 1 {1-\alpha}}. $$




\section{MLE of a bounded distribution}

Let $X$ be a random variable with p.d.f.~$\phi(x)$ with parameter(s)~$\theta$.

Let $Y_i \sim X$, $1 \le i \le n$, under the {\em bound condition} that $a \le Y_i \le b$.

P.d.f.~of $Y_i$ is $\phi_b(y) = 1/c \times \phi(x)$, where $c \eqdef \P (a \le X \le b)$.

Let $m_i$ be the multiplicity of $Y_i$.

$L = \prod_i \phi_b^{m_i}(x_i).$

$l = \sum_i m_i \log \phi_b(x_i) = \sum_i m_i (\log \phi(x_i) - \log c).$

$$0 = \frac d {d \theta} l = \sum_i m_i \frac d {d \theta} \log \phi(x_i) - \sum_i m_i \frac {\frac d {d \theta} c} c.$$

$$\mu_i \eqdef \frac {m_i} {\sum_j m_j}.$$

$$\sum_i \mu_i \frac d {d \theta} \log \phi(x_i) = \frac {\frac d {d \theta} c} c.$$



\section{Entropy}

\subsection{Preliminary definitions}
Let $X$ be a random variable with p.d.f.~$\phi(x)$.

The {\em support} of~$X$ is
$$\supp(X) \eqdef \supp(\phi) \eqdef \text{minimal event~$A$ of the $\sigma$-algebra of~$X$ s.t.}~\{x:\phi(x) > 0\} \subseteq A. $$

Let $Y \eqdef k X + a$, where $k > 0$.

Then the p.d.f.~of~$Y$ is $$\frac 1 k \ \phi \left(\frac {y - a} k \right). $$


\subsection{Location and scale parameters}
Let $X$ be a random variable with p.d.f.~$\phi(x|\mu,\sigma)$, where $\mu \in S_\mu$ and $\sigma \in S_\sigma \subseteq (0,\infty)$.

Let $\phi_{\mathrm {std}}(x) \eqdef \phi(x|0,1) $.

If
\begin{equation}
\phi(x|\mu,\sigma) = \frac 1 \sigma \phi_{\mathrm {std}}\left(\frac {x - \mu} \sigma \right)  \label{loc_scale}
\end{equation}
then $\mu$ is the {\em location} parameter of~$\phi$ and~$\sigma$ is the {\em scale} parameter of~$\phi$.

A p.d.f.~has location and scale parameters if $S_\mu \eqdef \{0\}$ and $S_\sigma \eqdef \{1\}$.

Let~$X_{\mathrm {std}}$ have the p.d.f.~$\phi_{\mathrm {std}}$,
then $X \eqdef \sigma X_{\mathrm {std}} + \mu$ has p.d.f.~with location parameter~$\mu$ and scale parameter~$\sigma$.


\subsection{Information and entropy}
Let~$X$ be a random variable with p.d.f.~$\phi(x)$.

Let {\em information of~$X$} be the random variable $I(X) \eqdef -\log \phi(X)$.
{\em Shannon information} uses $\log_2$.

The {\em entropy of~$X$} is $h(X) \eqdef \E I (X)$.

Let $\mathfrak X \eqdef \supp(X)$.

$$\E I(X) = - \int_{\mathfrak X} \phi(x) \log \phi(x) \ dx. $$
$$\var I(X) = \int_{\mathfrak X} \phi(x) \log^2 \phi(x) \ dx - (\E I(X))^2.$$

Since $I(X + a) \sim I(X)$,
$\E I(X + a) = \E I(X)$ and $\var I(X + a) = \var I(X)$.

Let $Y \eqdef k X$, where $k > 0$.
Let $\mathfrak Y \eqdef \supp(Y)$.
Then
\begin{equation}
\begin{split}
\E I(kX) &= - \int_{\mathfrak Y} \frac 1 k \phi\left(\frac y k\right) \log \left(\frac 1 k \phi\left(\frac y k\right)\right) dy  \\
      &= \log k - \frac 1 k \int_{\mathfrak Y} \phi\left(\frac y k\right) \log \phi\left(\frac y k\right) dy \\
      &= \log k - \frac 1 k \int_{\mathfrak X} \phi (x) \log \phi (x) \ k \ dx  \qquad (y = kx) \\
      &= \E I(X) + \log k 
\end{split}
\end{equation}
and
\begin{equation}
\begin{split}
\var I(kX) &= \int_{\mathfrak Y} {\frac 1 k} \phi\left({\frac y k}\right) \log^2 \left({\frac 1 k} \phi\left({\frac y k}\right)\right) dy 
              - (\E I(kX))^2 \\
  &= \int_{\mathfrak Y} {\frac 1 k} \phi\left({\frac y k}\right) \left( \log^2 \phi\left({\frac y k}\right) - 2 \log k \log \phi \left({\frac y k}\right) + \log^2 k \right) dy 
    - (\E I(kX))^2 \\
  &= \int_{\mathfrak Y} {\frac 1 k} \phi\left({\frac y k}\right) \log^2 \phi\left({\frac y k}\right) dy 
     - 2 \log k \int_{\mathfrak Y} {\frac 1 k} \phi\left({\frac y k}\right) \log \phi \left({\frac y k}\right) dy
     + \log^2 k 
     - (\E I(kX))^2  \\
  &= \int_{\mathfrak X} {\frac 1 k} \phi (x) \log^2 \phi(x) \ k \ dx 
     - 2 \log k \int_{\mathfrak X} {\frac 1 k} \phi(x) \log \phi (x) \ k \ dx
     + \log^2 k 
     - (\E I(kX))^2  \qquad (y = kx) \\
  &= \var I(X) + (\E I(X))^2
     + 2 \log k \ \E I(X)
     + \log^2 k 
     - ((\E I(X))^2 + 2 \log k \ \E I(X) + \log^2 k)  \\
  &= \var I(X).
\end{split}
\end{equation}

If $X = k X_{\mathrm {std}} + a$, where $k > 0$,
then
$$\E I(X) = \E I(X_{\mathrm {std}}) + \log k, $$
$$\var I(X) = \var I(X_{\mathrm {std}}).$$ 

Let~$\ve X$ and~$\ve Y$ be multivariate random variables and $\ve Y = \mat K \ve X$,
then $$ \phi_Y = \frac 1 {|\det \mat K|} \phi_X (\mat K^{-1} \ve Y), $$
$$ \E I(\mat K \ve X) = \E I(\ve X) + \log |\det \mat K|, $$
$$ \var I(\mat K \ve X) = \var I(\ve X). $$

If the components of~$\ve X$, denoted by $X_i$, where $1 \le i \le n$, are independent of each other,
then
$$ I(\ve X) = \sum_{i=1}^n I(X_i). $$


\subsection{Principle of maximum entropy}
For p.d.f.~$\phi$ and~$\psi$,
the {\em cross-entropy} of p.d.f.~$\psi$ w.r.t.~p.d.f.~$\phi$ is
$$ h(\phi,\psi) \eqdef - \int_{\supp(\phi)} \phi(x) \log \psi(x) \ dx, $$
\begin{equation} \label{min_entropy}
 \forall \phi, \psi : h(\phi, \phi) \le h(\phi,\psi),  
\end{equation}
If $h(\phi, \phi) = h(\phi,\psi)$
then the set $\{x: \phi(x)= \psi(x)\}$ has measure~1.
$$ \forall \phi, \psi : \supp(\phi) \supset \supp(\psi) \implies h(\phi,\psi) = \infty. $$
$$ h (\phi) \eqdef h(\phi,\phi). $$

If~$X$ is a random variable with p.d.f.~$\phi(x)$ then $h(X) = h(\phi)$.

Let~$\Phi(\theta)$ be a set of p.d.f.~with constraint parameters~$\ve \theta$.

Then the {\em maximum entropy p.d.f.~$\psi$ of constraint parameters~$\theta$}
$$ \psi(\theta) \eqdef \arg \max_{\phi \in \Phi(\theta)} h(\phi). $$ 
$$ \forall \psi \in \Phi(\theta) \
  [\exists f : \forall \phi \in \Phi(\theta): h(\phi,\psi) = f(\ve \theta)] \implies \psi = \psi(\theta).
$$
\proof{
$$ h(\psi) = f(\ve \theta) $$
and by~(\ref{min_entropy}).
}

\comm {\em principle of maximum entropy}
\comm {$\psi(\theta)$ vs. entropy test}

Examples of~$\Phi(\theta)$ and~$\psi(\theta)$:

\begin{enumerate}

\item
$ \Phi(\theta) = \{\phi: \supp(\phi)= \R, \E X(\phi)=\mu, \var X(\phi)=\sigma^2\}$, $\psi(\theta) = N(\mu,\sigma^2)$:
\begin{equation*}
\begin{split}
 h(\phi,\psi(\theta)) &= \int_\R \phi(x) \left(\log \sqrt {2 \pi} \sigma + \frac 1 2 \left(\frac {x - \mu} \sigma \right)^2 \right) \ dx \\
  &= \log \sqrt {2 \pi} \sigma + \frac 1 {2 \sigma^2} \int_\R \phi(x) (x - \mu)^2 \ dx \\
  &= \log \sqrt {2 \pi} \sigma + \frac 1 {2 \sigma^2} \sigma^2 \\
  &= \log \sqrt {2 \pi} + \log \sigma + \frac 1 2 \\
  &= \frac 1 2 (1 + \log 2 \pi) + \log \sigma.
\end{split}
\end{equation*}
$$ \frac 1 2 (1 + \log 2 \pi) \approx 1.42. $$

\item
$ \Phi(\theta) = \{\phi: \supp(\phi)= \R_+, \E X(\phi)=\mu\}$, $\psi(\theta) = \mathit {Exponential}(1 / \mu)$:
\begin{equation*}
\begin{split}
 h(\phi,\psi(\theta)) &= \int_{\R_+} \phi(x) \left (\log \mu + \frac x \mu \right) \ dx \\
  &= \log \mu + \frac 1 \mu \int_{\R_+} \phi(x) x \ dx \\
  &= \log \mu + \frac 1 \mu \mu \\
  &= 1 + \log \mu.
\end{split}
\end{equation*}

\item 
$ \Phi(\theta) = \{\phi: \supp(\phi)= [x_{\mathrm {min}}, x_{\mathrm{max}}], x_{\mathrm {min}} \ge 0, \E (\log X(\phi))=\lambda\}$, 
$\psi(\theta) = Power(x_{\mathrm {min}}, x_{\mathrm{max}},\gamma(\lambda))$:
\begin{equation*}
\begin{split}
 h(\phi,\psi(\theta)) &= 
     - \int_{x_{\mathrm {min}}}^{x_{\mathrm {max}}} \phi(x) \left(\log c \gamma - (\gamma + 1) \log x \right) \ dx \\
  &= - \log c \gamma + (\gamma + 1) \lambda.
\end{split}
\end{equation*}

\end{enumerate}


\subsection{Mixture of distributions}
Let~$X$ be a mixture of distributions with the p.d.f.
$$ \phi_x (x) = \sum_y \phi_y (y) \phi_{x|y}(x|y), $$
then
\begin{equation}
\begin{split}
  \E I(X) &= - \int_{\mathfrak X} \phi_x(x) \ \log \phi_x(x) \ d x \\
          &= - \int_{\mathfrak X} \left(\sum_y \phi_y (y) \phi_{x|y}(x|y) \right) \ \log \left(\sum_y \phi_y (y) \phi_{x|y}(x|y) \right) \ d x \\
          &= - \sum_y \phi_y (y) \int_{\mathfrak X} \phi_{x|y}(x|y) \ \log \left(\sum_z \phi_y (z) \phi_{x|y}(x|z) \right) \ d x \\
          &\ge - \sum_y \phi_y (y) \int_{\mathfrak X} \phi_{x|y}(x|y) \ \log \left(\phi_y (y) \phi_{x|y}(x|y) \right) \ d x \\
          &= h(\phi_y) + \E h(\phi_{x|y}|Y),
\end{split}
\end{equation}
and the inequality is tight.


\subsection{Conditional distribution}
Let~$\phi(x,y)$ be a p.d.f.,
then
$$ \phi(x,y) = \phi_x(x) \phi_{y|x}(y|x), $$
\begin{equation}
\begin{split}
h(\phi) &= -\int_{\mathfrak X} \int_{\mathfrak Y} \phi(x,y) \ \log \phi(x,y) \ dx \ dy \\
        &= -\int_{\mathfrak X} \int_{\mathfrak Y} \phi_x(x) \ \phi_{y|x}(y|x) \ (\log \phi_x(x) + \log \phi_{y|x}(y|x)) \ dx \ dy \\
        &= h(\phi_x) + \int_{\mathfrak X} \phi_x(x) \ h(\phi_{y|x}|x) \ dx \\
        &= h(\phi_x) + \E h(\phi_{y|x}|X).
\end{split}
\end{equation}

\begin{equation}
\phi(x,y) = \phi_x(x) \ \phi_y(y) \equivalent h(\phi) = h(\phi_x) + h(\phi_y),  \label{dep_h}
\end{equation}
where the RHS is equivalent to
$$ h(\phi_y) = \E h(\phi_{y|x}|X). $$

Proof of the necessity of~(\ref{dep_h}): \par
since
\begin{equation}
\begin{split}
 h(\phi_y) &= - \int_{\mathfrak Y} \left(\int_{\mathfrak X} \phi_x(x) \ \phi_{y|x}(y|x) \ dx \right) \log \phi_y(y) \ dy \\
           &= \int_{\mathfrak X} \phi_x(x) \left( - \int_{\mathfrak Y} \phi_{y|x}(y|x) \log \phi_y(y) \ dy \right) dx \\
           &= \int_{\mathfrak X} \phi_x(x) \ h(\phi_{y|x}, \phi_y|x) \ dx,
\end{split}
\end{equation}
$$ \forall x :  h(\phi_{y|x}, \phi_y|x) \ge h(\phi_{y|x}|x) $$
and we assume that
$$ \int_{\mathfrak X} \phi_x(x) \ h(\phi_{y|x}, \phi_y|x) \ dx = \int_{\mathfrak X} \phi_x(x) \ h(\phi_{y|x}|x) \ dx, $$
the set $\{x: h(\phi_{y|x}, \phi_y|x) = h(\phi_{y|x}|x) \}$
has measure~1 and for these~$x$ the set $\{y: \phi_y(y) = \phi_{y|x}(y|x)\}$ has measure~1,
so $\phi(x,y) = \phi_x(x) \ \phi_y(y)$ on the set of $(x,y)$ of measure~1.
\QED


\subsection{Likelihood}
Let $X_1, X_2, ..., X_n \sim X$ be independent random variables with p.d.f.~$\phi$.

For a p.d.f.~$\psi$ 
$$ \hat h(\psi) \eqdef - \frac 1 n \sum_{i=1}^n \log \psi(X_i) $$
is the negative average loglikelihood of~$\psi$.

$$ \E \hat h(\psi) = h(\phi,\psi). $$
$$ \lim_{n \to \infty} \var \hat h(\psi) = 0. $$

$$ \hat h (\phi) = \frac 1 n \ \sum_{i=1}^n I(X_i). $$


\begin{enumerate}
\item
Let~$\Psi$ be a set of distributions.

$$ \hat \phi(\Psi) \eqdef \arg \min_{\psi \in \Psi} \hat h(\psi) $$
is the MLE of~$\phi$ in~$\Psi$.

If~$\phi \in \Psi$ then $\hat \phi(\Psi)$ is a consistent estimate of~$\phi$:
$$ \lim_{n \to \infty} \hat \phi(\Psi) = \arg \min_{\psi \in \Psi} \lim_{n \to \infty} \hat h(\psi) = \arg \min_{\psi \in \Psi} h(\phi,\psi) = \phi $$
by~(\ref{min_entropy}).

\item
The {\em fitness} hypothesis 
$$ H_0: \phi = \psi. $$

If~$H_0$ is true then 
$$ \E \hat h(\psi) = \E I (X), $$
$$ \var \hat h(\psi) = \frac 1 n \ \var I(X), $$
and by the central limit theorem
$$\lim_{n \to \infty} \hat h(\psi) \sim \mathit{Normal} \left(\E I (X), \frac 1 n \ \var I(X) \right).$$
For small~$n$ the bounds on the tails of the p.d.f.~of $\hat h(\psi)$ may be defined by the Chebyshev inequality.

\item
If $\psi \eqdef \hat \phi(\Psi)$ then 
the fitness hypothesis is rejected if $\hat h(\psi)$ is too large or too small w.r.t.~$\hat h(\psi)$.

This is the {\em entropy MLE fitness test}.

\item 
Let $\psi$~be the p.d.f.~of $$\mathit{Normal} \left(\bar X, \frac 1 n \sum_{i=1}^n (X_i - \bar X)^2 \right), $$
then $h(\psi) = \hat h(\psi)$.

If $\psi = \hat \phi(\Psi)$ then the p-value of the entropy MLE fitness test is~1.

\item
Let $\psi \in \Psi$.
For this pair of hypotheses
\begin{equation*}
\begin{split}
H_0: \phi &= \psi \\
H_1: \phi &= \hat \phi(\Psi) 
\end{split}
\end{equation*}
the p-value is $\P \left(\hat h(\psi) \le \hat h \left(\hat \phi(\Psi) \right) \Big| H_0 \right)$,
which defines an {\em entropy confidence set} of~$\psi$ for which~$H_0$ is not rejected.


\end{enumerate}




\Questions
\begin{itemize}
  \item Existence of $\supp(X)$.
  \item Existence of $\E I(X)$ and $\var I(X)$ 
  \item Kullback-Leibler divergence
  \item Principle of maximum entropy
\end{itemize}


\section{Outliers}

A numeric variable is assumed to have the p.d.f.
\begin{equation*}
  \phi(x) \eqdef 
    \begin{cases}
       \phi_N(x), &\text{ if } x \le t\\
       (1 - \Phi_N(t)) \ \phi_U(x),   &\text{ else,}
     \end{cases}
\end{equation*}
where $\phi_N$ is a p.d.f.~of a normal distribution,
and $\phi_U$ is a p.d.f.~of a uniform distribution whose domain belongs to $[t,\infty]$.

Then for data $\{x_i \mid 1 \le i \le n\}$ 
the likelihood
$$ L(t,N,U) = \prod_{x_i \le t} \phi_N(x_i) \ \prod_{x_i > t} (1 - \Phi_N(t)) \ \phi_U(x_i) $$
and 
$$ l(t,N,U) = - \log L(t,N,U) = -\sum_{x_i \le t} \log \phi_N(x_i) - \sum_{x_i > t} \log [(1 - \Phi_N(t)) \ \phi_U(x_i)]. $$

The outliers are $\{x_i \mid x_i > \arg_t \min l(t,N,U)\}$.

$$ m \eqdef |\{x_i \mid x_i \le t\}|. $$

$$ \min_N \frac 1 m \left(- \sum_{x_i \le t} \log \phi_N(x_i) \right) = \frac 1 2 \left(1 + \log 2 \pi + \log \sigma^2 \right), $$
where
$$ \sigma^2 \eqdef \var \{x_i \mid x_i \le t\}. $$

$$ \min_U \left(- \sum_{x_i > t} \log \phi_U(x_i) \right) = (n - m) \log \ (\max_i x_i - t). $$

Therefore,
$$ \min_{N,U} l(t,N,U) = \frac m 2 \left(1 + \log 2 \pi + \log \sigma^2 \right) + (n - m) \ \log \frac {\max_i x_i - t} {1 - \Phi_N(t)}. $$

The value $\arg_t \min l(t,N,U)$ can be found in time~$O(n \log n)$ by sorting the data and scanning it once.

This procedure is not idempotent.


\section {Prediction of a nominal variable}
Let $A \eqdef \{a_0, a_1, \dots, a_{n-1}\}$ be a set of {\em actions}.

Let~$X$ be a nominal variable taking on values $\{x_0, x_1, \dots, x_{m-1}\}$.

Let $l: A, X \mapsto R$ be a {\em loss} function.

Let~$D$ be data.

The goal is to select such an action~$a_i$ given~$D$ that $\E l(a_i, X|D)$ is minimized, 
which will be referred to as {\em the best} action.
(There may be $>1$ best actions.)

$$ l(a_i|D) \eqdef \E l(a_i, X|D) = \sum_j l(a_i,x_j) \P (X=x_j|D). $$
The best action 
$$ a(D) \eqdef \arg\min_i l(a_i|D). $$
$$ a(x_j) \eqdef \arg \min_i l(a_i,x_j). $$

Let the {\em risk} of action~$a_i$ given $X=x_j$ be defined as
$$ r(a_i,x_j) \eqdef l(a_i,x_j) - l(a(x_j),x_j). $$
$$ \forall a_i, x_j : r(a_i,x_j) \ge 0. $$
The {\em expected risk} 
$$ r(a_i|D) \eqdef  \E r(a_i, X|D) = \sum_j r(a_i,x_j) \P (X=x_j|D). $$
$$ l_{min}(D) \eqdef \sum_j l(a(x_j),x_j) \P (X=x_j|D). $$
Then
$$ l(a_i|D) = \sum_j (r(a_i,x_j) + l(a(x_j),x_j)) \P (X=x_j|D) = r(a_i|D) + l_{min}(D), $$
and
$$ a(D) = \arg \min_i r(a_i|D). $$

If $n = m = 2$, $r(a_0,x_0) = r(a_1,x_1) = 0$, $r_0 \eqdef r(a_1,x_0)$ and $r_1 \eqdef r(a_0,x_1)$ then
$$ r(a_i|D) = r_{1-i} \ \P (X = x_{1-i}|D), \text{ for } i \in \{0,1\}. $$
Both $r(a_0|D)$ and $r(a_1|D)$ are functions of $\P(X=x_0|D)$ 
and have graphics in the form of line segments $[(0,r_1),(1,0)]$ and $[(0,0),(1,r_0)]$ respectfully.
Then $a(D)$ is the action whose line segment is lower given $\P(X=x_0|D)$.


\subsection {Communication system}
The random variables~$X$ and~$D$ are a {\em communication system} iff 
$$ \P(\max_i \P(X=x_i|D) \approx 1) \approx 1 $$
and the computation of $\arg \max_i \P(X=x_i|D)$ is fast.

By Bayes' theorem,
$$ \P(X=x_i|d) = \frac {\P(d|X=x_i) \P(X=x_i)} {\sum_j \P(d|X=x_j) \P(X=x_j)}. $$

The computation of $\arg \max_i \P(X=x_i|D)$ is fast if
$$ \forall i \ \exists p_i \in Prob : \P\left[\arg \max_i \P(X=x_i|D) = \{i: \P(d|X=x_i) \ge p_i\} \right] \approx 1. $$

Examples of communication systems:
\begin{itemize}
\item Natural language: $D$ is text.
\item Biology: $D$ is a protein, $X$ is a protein function.
\end{itemize}

\begin{equation*}
\begin{split}
 \alpha \le \P(X=x_i|d) = \frac {\P(d|X=x_i) \P(X=x_i)} {\P(d)} &\equivalent \frac {\P(d|X=x_i)} {\P(d)} \ge  \frac \alpha {\P(X=x_i)} \\
   &\equivalent I(d) - I(d|X=x_i) \ge \log \alpha + I(X=x_i).
\end{split}
\end{equation*}


\section{Prediction of a boolean variable}

Let~$X \sim \mathit{Bernoulli}(p)$.

Let~$\hat X(\theta) \in \{0,1\}$ be the prediction of~$X$ made by method~$\theta \in \Theta$.

If $X=1$ is interpreted as a {\em dangerous signal} 
then the prediction situations can be classified into 4~groups:
\begin{center}
\begin{tabular}{|c||c|c|}
\hline       & noise       & signal \\
\hline 
\hline Ok    & correct (True Negatives)     & miss  (False Negatives)\\
\hline alarm & false alarm (False Positives)& correct (True Positives) \\
\hline
\end{tabular}
\end{center}

Let~$r : \{0,1\}, \{0,1\} \mapsto R_+$ be the risk function, 
where the first  argument is the value of~$\hat X$
and   the second argument is the value of~$X$,
defined by the table:
\comm{
\begin{equation*}
r(x,y) \eqdef 
  \begin{cases}
    r_0, \text{ if } x = 1, y = 0\\
    r_1, \text{ if } x = 0, y = 1\\
    0, \text{ else.}
  \end{cases}
\end{equation*}
}

\begin{center}
\begin{tabular}{|c||c|c|}
\hline              & $X=0$ & $X=1$ \\
\hline 
\hline $\hat X = 0$ & 0     & $r_1$ \\
\hline $\hat X = 1$ & $r_0$ & 0 \\
\hline
\end{tabular}
\comm {Higher rows}
\end{center}

Let 
$$e_0(\theta) \eqdef \P(\hat X(\theta) = 1 | X = 0),$$
$$e_1(\theta) \eqdef \P(\hat X(\theta) = 0 | X = 1)$$
be the probabilities of {\em type~I} and {\em type~II} errors respectfully. 

Then the expected risk 
\begin{equation} \label{expectedRisk}
\E r(\hat X(\theta), X) = r_0 (1-p) \ e_0(\theta) + r_1 p \ e_1(\theta). 
\end{equation}
Let $\theta_1 \preceq_{p,r} \theta_2$ iff method~$\theta_1$ is {\em better} than method~$\theta_2$ given~$p$ and the risk function~$r$:
$$ \theta_1 \preceq_{p,r} \theta_2 \equiv \E r(\hat X(\theta_1), X) \le \E r(\hat X(\theta_2), X). $$

For $i \in \{0,1\}$ let~$\vartheta_i$ be the two trivial prediction methods such that $\hat X(\vartheta_i) = i$ always,
then
$$ e_i(\vartheta_{1-i}) = 1, $$
$$ e_{1-i}(\vartheta_i) = 0. $$

Suppose $\{\vartheta_0, \vartheta_1\} \subseteq \Theta$.

Let
$$ROC(\Theta) \eqdef \{(e_0(\theta), e_1(\theta)) : \theta \in \Theta\}$$
be the {\em receiver operating characteristic} of~$\Theta$.

Let
$$ ROC^*(\Theta) \eqdef \{(x, y) : (x,y) \in \mathrm {ConvexHull}(ROC(\Theta) \cup \{(1,1)\}) \And x + y \le 1 \}. $$
$$ \Theta^* \eqdef \{\theta \in \Theta : (e_1(\theta),e_2(\theta)) \in ROC^*(\Theta)\}.
$$

Then 
$$ \forall p \ \forall \ r \ \forall \ \theta \in \Theta \ \exists \theta^* \in \Theta^* : \theta^* \preceq_{p,r} \theta. $$
In other words, to find the best~$\theta \in \Theta$ it is enough to consider~$\Theta^*$.

$\Theta^*$ defines a 1-1 relationship between $e_0(\theta)$ and $e_1(\theta)$, where $\theta \in \Theta^*$,
so that~$\theta$ is identified by~$e_0(\theta)$.

$\Theta$ may be optimized so that $\Theta = \Theta^*$.
This may occur in biological systems due to evolution or in technical systems due to improvement.
If all systems share the same~$\Theta$, then $\theta \in \Theta$ is identified by~$e_0(\theta)$.

Let $e_1: Prob \mapsto Prob$ such that
$$ e_1(x) \eqdef \text{the unique element of } \{ y : (x,y) \in ROC^*(\Theta) \}. $$
The function $e_1(x)$ is decreasing in~$x$.
$$ \forall x \in Prob : x + e_1(x) \le 1. $$


\subsection {p-Value}
Let~$X$ and data~$D$ have a joint distribution.
Let~$\phi(D)$ be the p.d.f.~of~$D$.

The {\em p-value} of~$d$
$$ p(d) \eqdef \P\left[\phi(D|X=0) \le \phi(d|X=0) \ | \ X=0 \right]. $$
$$ p(d) \in Prob. $$
$$ p(D) \sim U(0,1). $$

If $\theta = p$ and $\hat X(p) = 1 \equiv p(D) \le p$ then $e_0(p) = p$.

If, in addition, $r_1=0$ and $r_0=1$ then $\E r(\hat X(p), X) \le p$ by~(\ref{expectedRisk}).


\subsection {Upper bound on expected risk}
$$ r_{max} \eqdef \max_{\theta \in \Theta^*} \{r_0 \ e_0(\theta), r_1 \ e_1(\theta) \}. $$
$$ \forall \theta \in \Theta^* : \E r(\hat X(\theta), X) \le r_{max}, $$
which is a tight bound.

$$r_{max} \le \max_{x \in Prob} \{ r_0 x, \ r_1 e_1(x) \}. $$

Let~$e_0^* \in Prob$ be such that $r_0 e_0^* = r_1 e_1(e_0^*)$, then
$$ \max_{x \in Prob} \{ r_0 x, r_1 e_1(x) \} = r_0 e_0^*. $$
$$ e_0^* + e_1(e_0^*) \le 1. $$
$$ \frac {e_0^* + e_1(e_0^*)} {e_0^*} \le \frac 1 {e_0^*}. $$
$$ 1 + \frac {r_0} {r_1} \le  \frac 1 {e_0^*}. $$
$$ e_0^* \le \frac {r_1} {r_0 + r_1}. $$
Similarly,
$$ e_1^* \le \frac {r_0} {r_0 + r_1}. $$
Then
$$ \forall \theta \in \Theta^* : \E r(\hat X(\theta), X) \le r_0 e_0^* \le \frac {r_0 r_1} {r_0 + r_1}. $$


\section{Bonferroni correction}

Let~$X_i$, where $1 \le i \le n$, be the p-value of test~$i$ of the same~$H_0$.
Suppose all tests are independent, and $H_0$ holds.
Then $X_i \sim U(0,1)$.

Let~$X_{\min} \eqdef \min_i X_i$.

If $x \approx 0$ then
$$ \P(X_{\min} \le x) = 1 - \prod_{i=1}^n \P(X_i \ge x) = 1 - (1 - x)^n \approx 1 - e^{-xn} \approx 1 - (1 - xn) = xn, $$
which is the {\em E-value}.



\section{Prediction}

Let~$\ve X \in \R^p$ be a random vector of {\em predictor} variables and $Y \in \R$ be a {\em target} variable.
The goal is to identify the conditional probability $\P(Y=y|\ve X = \ve x) = \phi_Y(y|\ve x) \ dy$.
A {\em prediction} of~$Y$ given~$\ve x$ is a function~$\hat y(\ve x)$.
A prediction~$\hat y(\ve x)$ is a {\em regression} iff $\hat y(\ve x) = \E (Y|\ve X = \ve x)$.

Let i.i.d.~$(\ve X_i, Y_i)$, where $1 \le i \le n$, be random {\em data} with the p.d.f.~$\phi(\ve x,y) = \phi_Y(y|\ve x) \phi_X(\ve x)$.
The negative average loglikelihood of a conditional p.d.f.~$\psi(y|\ve x)$ is
$$ \hat h(\psi) \eqdef - \frac 1 n \sum_{i=1}^n \log \psi(Y_i|\ve X_i). $$
$$ \E \hat h(\psi) = - \int \phi(\ve x,y) \log \psi(y_i|\ve x_i) \ d \ve x \ d y 
     =   \int h(\phi_Y, \psi | \ve x) \ \phi_X(\ve x) \ d \ve x
     \ge \int h(\phi_Y, \phi_Y | \ve x) \ \phi_X(\ve x) \ d \ve x.
$$
$$ \lim_{n \to \infty} \var \hat h(\psi) = 0. $$

The function~$\hat h(\psi)$ is the minimization criterion.


\subsection{Boolean target}

Let~$Y \in \{0,1\}$.

The prediction 
$$ \hat y(\ve x) \eqdef \P(Y|\ve X = \ve x) $$
is a regression.
 
\begin{equation*}
\begin{split}
\P(Y|\ve X = \ve x) 
  = \frac {\P(\ve X = \ve x|Y) \P(Y)} {\P(\ve X = \ve x|Y) \P(Y) + \P(\ve X = \ve x|\neg Y) \P(\neg Y)} 
  = \frac {\frac {\P(\ve X = \ve x|Y) \P(Y)} 
                  {\P(\ve X = \ve x|\neg Y) \P(\neg Y)}
           }
           {\frac {\P(\ve X = \ve x|Y) \P(Y)} 
                  {\P(\ve X = \ve x|\neg Y) \P(\neg Y)}
            + 1
           }.
\end{split}
\end{equation*}
$$ z(\ve x) \eqdef   \log \frac {\P(\ve X = \ve x|Y)} 
                                {\P(\ve X = \ve x|\neg Y)}
                   + \log \frac {\P(Y)} {\P(\neg Y)}. 
$$

Let
$$ F(x) \eqdef \frac {e^x} {e^x + 1} $$
be the c.d.f.~of $Logistic(0,1)$. 
Then
$$ \P(Y|\ve X = \ve x) = F(z(\ve x)). $$

If $X \sim Logistic(0,1)$ then 
$$ Y = (X \le z(\ve x)).$$

$$ \P(\neg Y|\ve X = \ve x) = F(- z(\ve x)). $$

The likelihood
$$ L = \prod_i \P(Y|\ve x_i)^{y_i} \left(1 - \P(Y|\ve x_i) \right)^{1-y_i}. $$
$$ \hat h (\psi) = - \frac 1 n \log L = \frac 1 n \sum_i \left[ \log \left(e^{z(\ve x_i)} + 1\right) - y_i z(\ve x_i) \right]. $$

The geometric mean of the correct prediction is $\sqrt[n]L = e^{- \hat h(\psi)}$.

Let~$\theta$ be a parameter of~$z(\ve x)$ then the MLE of~$\theta$ is the solution to
\begin{equation}\label{boolPredictMLE}
0 = \frac d {d \theta} \hat h (\psi) = \frac 1 n \sum_i (\hat y(\ve x_i) - y_i) \frac d {d \theta} z(\ve x_i). 
\end{equation}

\comm{
    If $z(\ve x)$ is monotone w.r.t.~$\theta$ then $\hat y(\ve x) = F(z(\ve x))$ is monotone w.r.t.~$\theta$ and~(\ref{boolPredictMLE}) can be solved by binary search. 
}

If $Y = (z_0(\ve x) > 0)$, i.e., $z_0(\ve x) = 0$ defines a surface in the space of~$\ve X$ separating $\{\ve x \mid y = 0\}$ from $\{\ve x \mid y = 1\}$,
then any $z(\ve x) \eqdef k \ z_0(\ve x)$, where $k > 1$, increases~$L$ in comparison with $z(\ve x) \eqdef z_0(\ve x)$,
and the optimization of~$\theta$ does not end.



\subsubsection{Linear $z(\ve x)$}

If
\begin{equation}\label{linZ}
z(\ve x) = \ve \beta' \ve x
\end{equation}
then by~(\ref{boolPredictMLE})
$$ \frac d {d \beta_j} \hat h (\psi) = \frac 1 n \sum_i (\hat y(\ve x_i) - y_i) x_{ij}. $$

The Hessian of~$\hat h (\psi)$
$$ \mat H \eqdef \frac {d^2} {d \beta_j d \beta_k} \hat h (\psi) = \frac 1 n \sum_i \hat y (\ve x_i) (1 - \hat y (\ve x_i)) \ x_{ij} \ x_{ik}. $$
$$ \mat H \succeq 0. $$
\proof {
$\mat H = \frac 1 n \mat X' \ve \lambda^d \mat X$,
where $\mat X$ is the data with the elements~$x_{ij}$, 
and $\ve \lambda$ is the vector with the elements $\lambda_i = \hat y (\ve x_i) (1 - \hat y (\ve x_i))$.
}

Therefore, the function $\hat h (\psi)$ is convex.

If $X_j$, where $1 \le j < p$, are independent Boolean variables and $X_p = 1$,
then equation~(\ref{linZ}) is true, where
$$ \beta_j \eqdef w_j(1) - w_j(0), $$
\begin{equation}
w_j(x) \eqdef
\begin{cases}
\log \frac {\P(X_j = x|Y)} {\P(X_j = x|\neg Y)} &\text{ if } j < p, \\
\sum_{j < p} w_j(0) + \log \frac {\P(Y)} {\P(\neg Y)}, &\text{ if } j = p.
\end{cases}
\end{equation}
Equation~(\ref{linZ}) is also true if some $X_j$ are equal or opposite to each other.

% instability of solution if $\max {x_{.j} | y = \neg a} < \min {x_{.j} | y = a}$

If 
$\ve X|Y      \sim N(\ve \mu_1, \mat \Sigma)$ and 
$\ve X|\neg Y \sim N(\ve \mu_2, \mat \Sigma)$
then
$$ z(\ve x) = (\ve \mu_2 - \ve \mu_1)' \mat \Sigma^{-1} (\ve \mu_1 + \ve \mu_2 - 2 \ve x) + \log \frac {\P(Y)} {\P(\neg Y)}. $$



\end{document}


