\documentclass[10pt,a4paper]{article}
\usepackage[latin1]{inputenc}
\usepackage{amsmath}
\usepackage{amsfonts}
\usepackage{amssymb}
\usepackage[]{theorem}
\usepackage{hyperref}
%\usepackage{graphics,graphicx}
%\usepackage{pstricks,pst-node,pst-tree}
\usepackage[nottoc]{tocbibind}


% Default margins are too wide all the way around. I reset them here
\setlength{\topmargin}{-.5in}
\setlength{\textheight}{9in}
\setlength{\oddsidemargin}{.125in}
\setlength{\textwidth}{6.25in}

\theoremstyle{plain} \newtheorem{Lem}{Lemma}


\newcommand {\subsubsubsection} [1] {\vskip 0.5em {\bf \em #1}}

%\newcommand {\comm} [1] {}
\newcommand {\comm} [1]
{
\par
\setlength{\leftskip}{-0.8in}
\begin{tabular} {p{0.1cm} | p{\textwidth}}
  {\bf ?} & {#1}
\end{tabular}
\par
\setlength{\leftskip}{0in}
}


\newcommand {\cond} {, & \textrm{if }}
\newcommand {\otherwise} {, & \textrm{else}}

\newcommand {\Forall} {\forall \ }
\newcommand {\eqdef} {\triangleq}
\newcommand {\equivalent} {\Leftrightarrow}
\renewcommand {\implies} {\Rightarrow}
\renewcommand {\impliedby} {\Leftarrow}
\renewcommand {\And}  {\ \& \ }
\newcommand {\Or}  {\ \vee \ }
\newcommand {\QED} {$\square$ \ \par }
\newcommand {\proofword}  {{\par \sf Proof:\ }}
\newcommand {\proof} [1] {\proofword #1 \QED \par}
\newcommand {\definition} [2] {\bigskip {\bf Definition} {\rm \sc #1} \par #2 \QED \bigskip}
\newcommand {\example} [2] {\bigskip {\bf Example.} {#1} \par #2 \QED}
\newcommand {\theorem} [2] {\bigskip {\bf Theorem} {\rm \sc #1} \par #2 \QED \bigskip}
\newcommand {\ve} {\mathbf}
\newcommand {\vecc} {\boldsymbol}
\newcommand {\Questions} {\vspace{5mm} {\bf Unresolved Questions:}}
\newcommand {\assert} {\square \ }

\renewcommand {\P} {\mathrm P}
\newcommand {\E} {\mathrm E \ }
\newcommand {\var} {\mathrm {var} \ }
\newcommand {\cov} {\mathrm {cov} }
\newcommand {\cor} {\mathrm {cor} }
\newcommand {\N} {\mathbb N}
\newcommand {\Z} {\mathrm Z}
\newcommand {\R} {\mathbb R}
\newcommand {\C} {\mathbb C}
\newcommand {\I} {\mathbb I}
\newcommand {\supp} {\mathrm {Supp}}
\newcommand {\mat} {\mathrm}
\newcommand {\trc} {\mathrm {tr} \ }
\newcommand {\VC} {\mathrm {VC} }
\newcommand {\const} {\mathrm {const} }
\newcommand {\sgn} {\mathrm {sgn} }
\newcommand {\dom} {\mathrm {dom} }
\newcommand {\range} {\mathrm {range} }
\newcommand {\Arg} {\mathrm {Arg} \ }

\newcommand {\LCA} {\mathrm {LCA} }  % least common ancestor (in a tree)

\newcommand {\nul} {\mathrm {null}}
\newcommand {\Dt} {\Delta t}
\newcommand {\PC} {\mathrm {PC} }
\newcommand {\MDS} {\mathrm {MDS} }
\newcommand {\CC} {\mathrm {CC} }

\newcommand {\re} {\mathrm {re}}
\newcommand {\cl} {\mathrm {cl}}
\newcommand {\intr} {\mathrm {int}}

\newcommand {\<} {\langle}
\renewcommand {\>} {\rangle}

\newcommand {\dotve} [1] {\dot {\ve #1}}

\newcommand {\la} [1] {\lambda (#1) \ }

\newcommand {\lhs} {\mathrm {lhs}}
\newcommand {\rhs} {\mathrm {rhs}}

\providecommand{\keywords}[1]
{
  \small
  \textbf{\textit{Keywords---}} #1
}

\makeatletter
\DeclareRobustCommand{\cev}[1]{%
  {\mathpalette\do@cev{#1}}%
}
\newcommand{\do@cev}[2]{%
  \vbox{\offinterlineskip
    \sbox\z@{$\m@th#1 x$}%
    \ialign{##\cr
      \hidewidth\reflectbox{$\m@th#1\vec{}\mkern4mu$}\hidewidth\cr
      \noalign{\kern-\ht\z@}
      $\m@th#1#2$\cr
    }%
  }%
}
\makeatother





\title{Phylogeny of Pan-Genome}
\author{Vyacheslav Brover}


\begin{document}
\maketitle

\tableofcontents


\section{Notation}
The names of sets and the names of functions returning sets start with a capital letter, otherwise names start with a small letter.

The same name can be used as the name of a set and a name of different functions. 
In the latter case the functions are distinguished by the types of their arguments.

A pair of names where one name is with a star and the other name is without a star, e.g. $Name^*$ and $Name$, means that $Name^*$ is not observed, but $Name$ is observed.

The sign $\eqdef$ means "equal by definition".

$ Prob \eqdef [0,1] $ --- probability.

\begin{equation*}
\neg^a p \eqdef
  \begin{cases}
    p,          &\text{ if } a = 0 \mod 2,\\
    1-p,        &\text{ if } a = 1 \mod 2.\\
  \end{cases}   
\end{equation*}
$$ \bar p \eqdef \neg p \eqdef \neg^1 p. $$
$$ \neg^b (\neg^a p) = \neg^{(a+b\mod 2)} p. $$
$$ (\neg^a p)' = (-1)^a p'. $$

If~$S$ is a set then $\iota S$ is any element of~$S$.

$$ A \bot B \equiv A \cap B \ne \emptyset. $$


\section {Genome model}

\begin{enumerate}

\item
Sets: $Locus$, $Gene$, $Genome$ ("isolates"), $Species$.

Genome is a set of chromosomes.

$ gene: Locus \mapsto Gene. $

$ species: Genome \mapsto Species. $


\item
$ Locus^*: Genome \mapsto \{0,1\}^{Locus}$ --- genome loci.
This is a random variable.

$ Locus: Genome \mapsto \{0,1\}^{Locus}$ --- genome loci obtained by a {\em genome annotation} procedure.
This is a random variable.

$$ \forall i_1, i_2 \in Genome : i_1 \ne i_2 \implies Locus^*(i_1) \cap Locus^*(i_2) = Locus(i_1) \cap Locus(i_2) = \emptyset. $$


\item
$p_{annot\_error}: Genome \mapsto Prob$ --- the type~I error of annotation, {\em annotation error}:

$$ p_{annot\_error}(i) \eqdef \frac{|Locus^*(i) \setminus Locus(i)|} {|Locus^*(i)|}. $$

Concerning type~II errors, assume
$$   \Forall l \in Locus(i) \setminus Locus^*(i) 
   \ \Forall l^* \in Locus^*(i)
   : gene(l) \ne gene(l^*)
   .
$$

$p_{annot\_error}: Species \mapsto Prob$ --- annotation error for a species, assuming
$$ \Forall i \in Genome: species(i) = s \implies p_{annot\_error}(i) = p_{annot\_error}(s). $$

\item
$\pi^*: Genome, Gene \mapsto \N$ --- true number of {\em paralogs}, true {\em pan-genome} matrix:
$$\pi^* (i,g) \eqdef |\{l \in Locus: l \in Locus^*(i) \And gene(l) = g\}|. $$

Assume that in a genome all loci of the same gene have the same origin: by mutation of a single locus or by horizontal gene transfer.

All genomes of a species have a common phenotype.


$\pi: Genome, Gene \mapsto \N$ --- number of annotated {\em paralogs}, annotated {\em pan-genome} matrix:
$$\pi (i,g) \eqdef |\{l \in Locus: l \in Locus(i) \And gene(l) = g\}|. $$


For a genome~$i$ and for different loci~$l_1$ and~$l_2$ assume 
that the evens $l_1 \in Locus(i)$ and $l_2 \in Locus(i)$ are independent,
\comm {and that the annotation error is equal for all loci}
then
$$ \forall g \in Locus^*(i) \implies [\pi(i,g) \sim Binomial(\pi^*(i,g), 1 - p_{annot\_error})]. $$


\item
$Core : Species \mapsto \{0,1\}^{Gene}$ --- {\em core} set of genes, a minimal set of genes necessary for a genome to have the phenotype of the species.
This is a random variable.
Our goal is to estimate the core genes of a species.

$$   \Forall s \in Species 
   \ \Forall i \in Genome
   \ \Forall g \in Gene
   : species(i) = s \And g \in Core(s)
     \implies \pi^*(i,g) \ge 1. 
$$
$$ \Forall s_1, s_2 \in Species: s_1 \ne s_2 \implies Core(s_1) \not \subseteq Core(s_2). $$



\item
$\pi^*: Species, Gene \mapsto \N$ --- true number of paralogs of a gene in the core set of a species, assuming
$$   \Forall s \in Species 
   \ \Forall i \in Genome
   \ \Forall g \in Gene
   : species(i) = s \And g \in Core(s)
     \implies \pi^*(s,g) = \pi^*(i,g). 
$$
$$ \forall \ s : \pi^*(G) \sim \mathit{Zipf}(\alpha_{par}(s)). $$


\item
$core : Species, Gene \mapsto Prob$, defined as
$$ core (s,g) \eqdef \P(g \in Core(s)). $$
Consider a species~$s$.
$$ \E |Core(s)| = \sum_{g \in Gene} core(s,g), $$
Assume that for different genes~$g_1$ and $g_2$ the events $g_1 \in Core(s)$ and $g_2 \in Core(s)$ are independent, then
$$ \var |Core(s)| = \sum_{g \in Gene} core(s,g) (1 - core(s,g)). $$

\item
$Core: Genome \mapsto \{0,1\}^{Gene}$ --- observed core set of genes of a genome:
$$ Core(i) \eqdef \{gene(l): l \in Locus(i)\} \cap Core(species(i)). $$
$$ \E |Core(i)| = \sum_{g \in Gene: \pi(i,g) \ge 1} core(s,g), $$
$$ \var |Core(i)| = \sum_{g \in Gene: \pi(i,g) \ge 1} core(s,g) (1 - core(s,g)). $$

\item
$Accessory: Genome \mapsto \{0,1\}^{Gene}$ --- {\em accessory} set of genes of a genome:
$$Accessory(i) \eqdef \{gene(l): l \in Locus(i)\} \setminus Core(species(i)). $$
$$ \E |Accessory(i)| = \sum_{g \in Gene: \pi(i,g) \ge 1} (1-core(s,g)) = |Locus(i)| - \E |Core(i)|, $$
$$ \var |Accessory(i)| = \var |Core(i)|. $$


$p_{accessory}: Gene, Species \mapsto Prob$ defined as
$$ p_{accessory} (g,s) \eqdef \P(G = g | I \in Genome \And species(I) = s \And G \in Accessory(I)). $$
$$ \forall s \in Species: \sum_{g \in Gene} p_{accessory}(g,s)= 1. $$


\item
Let $S \subseteq Genome$.

$freq: Gene \mapsto \N$ --- absolute {\em gene frequency} in the set~$S$:
$$ freq(g) \eqdef \sum_{i \in S} \pi(i,g). $$

Suppose there is a species~$s$ such that $\Forall i \in S: species(i) = s$ and~$S$ is a random sample.

$$ \Forall g \in Gene: g \in Core(s) \implies freq(g) \sim Binomial(|S|\times  \pi^*(s,g), 1 - p_{annot\_error}(s)). $$

$$ \forall g \in Gene: g \not\in Core(s) \implies \hat p_{accessory}(g,s) = \frac {freq(g)} {\sum_{i \in S} |Accessory(i)|}, $$
where $\hat p_{accessory}(g,s)$ is the MLE estimate of $p_{accessory}(g,s)$.

$\alpha: Species \mapsto \R_{++}$.
$$ \Forall g \in Gene: g \not\in Core(s) \implies p_{accessory}(g,s) \sim Pareto(\alpha(s)). $$

\end{enumerate}


Core genes are unlikely to be predicted.


\Questions
\begin{itemize}
  \item $\alpha(s)$:
    \begin{itemize}
      \item Dependence of~$\hat \alpha(s)$ on sample size. $\hat \alpha(\texttt{E.~coli}) \ne \hat \alpha(\texttt{Prokaryotes})$.
      \item Make more accurate: use {\tt GENE.genomes = 1}; $\le 1$ and $> 5$ should go away.
      \item Fitness hypothesis is not rejected
      \item Estimate SD of MLE
      \item Explain the Zipf distribution of the subaccessory genes.
        \begin{itemize}
          \item Show that $\forall s: \alpha(s) = \alpha(s_0)$, where~$s_0$ is all prokaryotes.
          \item Biological mechanisms; explain the value of~$\alpha$.
        \end{itemize}
    \end{itemize}
\end{itemize}



\section{Phylogenetic tree data structure}

Let $Genes$ be a set of Boolean attributes.
$$ Species \eqdef 2^{Genes}. $$
Let $Genomes$ be a set of {\em genomes}.
$$ Genomes \subset Species. $$
$$ n \eqdef |Genomes|. $$
Let the {\em pan-genome} of $Genomes$ be their union.

A {\em pan-genome phylogenetic tree} of $Genomes$ is a directed tree $(Nodes,Arcs)$, where $$Genomes \subset Nodes \subset Species$$ and $Genomes$ are leaves.

Let the arcs in~$Arcs$ be directed towards the root.


\subsection{Construction of the tree}

Steps:
\begin{enumerate}
\item Construct an unrooted tree by the maximum parsimony criterion:
  \begin{itemize}
    \item Hierarchical clustering of genomes:
	  \begin{itemize}
	    \item Decompose the gene frequency distribution into {\em subaccessory}, {\em subcore} and {\em intersection core} distributions.
	          Subaccessory genes have a Zipf distribution and the other genes have Binomial distributions, 
	          where the intersection core has the distribution $Binomial(n,1-p)$ with $p$ being the {\em annotation error}.
	    \item MDS based on distances defined by subcore genes.
	    \item Decompose genomes into clusters by the decomposition of multidimensional normal distributions.
	    \item Repeat the above steps for each the obtained cluster of genomes.
	  \end{itemize}
   \item Locally optimize the tree by the maximum parsimony criterion.
   \end{itemize}
\item Find the root and its core set.
\item Locally optimize the rooted tree by the MLE criterion.
\end{enumerate}


\Questions
\begin{itemize}
  \item MDS dimensions: 
     \begin{itemize}
        \item Associate with genes.
        \item Interpret as a quantitative property of genomes.
     \end{itemize}
   \item Identification of the subcore genes to compute the distances for MDS: use only subcore genes with high peaks. 
         That will increase the ratio of between vs. within clusters scatter.
\end{itemize}



\subsection {Steiner tree problem}

Let $G = (V,E)$ be an undirected graph, where $V = \{0,1\}^{Gene}$ and $E = V^2$.

Let $S \eqdef \{Core(i): i \in Genome \}$. $S \subseteq V$.

For $(a,b) \in E$ let $len(a,b) \eqdef |Core(a) \setminus Core(b)| + |Core(b) \setminus Core(a)|$.

Then finding the maximum parsimony tree is the Steiner tree problem for $G$, $S$ and~$len$.
A solution to this problem is $(V^*, E^*)$, where $S \subseteq V^* \subseteq V$, $E^* \subseteq E$ and $V^* = V(E^*)$.
Let~$\mathfrak G$ be the set of all solutions. $\mathfrak G \ne \emptyset$.
$$ \forall (V^*,E^*) \in \mathfrak G : len(E^*) = const. $$

The function~$len$ is a distance.

The {\em Steiner vertices}
$$ Q \eqdef V^* \setminus S. $$

$$ \forall \ a \in Q : deg(a) \ge 3. $$
$$ |Q| \le |S| - 2. $$

For a given topology~$\mathfrak E$, the Sankoff algorithm finds a set~$Q$.
Therefore, the Steiner Tree problem can be solved for small~$|S|$ by examining all topologies.

For a solution $(V^*, E^*)$, let a {\em hyperedge} be the intersection of a maximal connected component of $(V^*, E^*)$, where each path traverses only the vertices in~$Q$, with~$S$.
Let $H(E^*) \subseteq \{0,1\}^S$ be the set of the hyperedges for the solution $(V^*, E^*)$.
Then
$$ \forall A \in H(E^*) : |A| \ge 2, $$
$$ \forall A,B \in H(E^*) : |A \cap B| \le 1. $$

For $(V^*, E^*) \in \mathfrak G$ 
$$ E_H(E^*) \eqdef \{(x,y) \in S^2 : x \ne y \And \exists A \in H(E^*) \ \{x,y\} \subseteq A\}. $$
$$ \forall A \in H(E^*) : A \text{ is a clique in } (S, E_H(E^*)). $$

A set $A \subseteq S$, where $A \ne \emptyset$, is a {\em Steiner component} iff
$$ \forall (V^*, E^*) \in \mathfrak G : A \text{ is connected in } (S, E_H(E^*)). $$
The trivial Steiner components are $\{\{x\}: x \in S\}$ and~$S$.
The set~$S$ can be partitioned into Steiner components.

For $A, B \subseteq V$ let
$$ len(A,B) \eqdef \min_{a \in A, b \in B} len(a,b). $$

Let the {\em diameter} of $A \subseteq V$ 
$$ diam(A) \eqdef \max_{x,y \in A} len(x,y). $$

Let the {\em sparseness} of $A \subseteq V$ 
$$ spar(A) \eqdef \max_{x \in A} \ len(\{x\}, A \setminus \{x\}). $$

Let $C \subseteq V$ be a {\em Steiner cluster} iff
$$ diam(C) < len(C, V \setminus C). $$

A set~$A$ is a Steiner component iff $A$ is a Steiner cluster.


\subsection{MLE criterion} \label{MLECriterion}

Let $g \in Genes$,  $(a, b) \in Arcs$.

For $s \in Species$ let 
\begin{equation*}
 c_g(s) \eqdef 
  \begin{cases}
    1, &\text{ if } g \in s,\\
    0, &\text{ else},
  \end{cases}
\end{equation*}
and let~$C_g(s)$ be the corresponding Boolean random variable.

Let the evolution model be
\begin{equation} \label{pq}
\begin{split}
p_g(a) &\eqdef \P (c_g(a) \mid c_g(b)), \\
q_g(a) &\eqdef \P (c_g(a) \mid \neg c_g(b)), 
\end{split}
\end{equation}
then
\begin{equation*} 
\P(C_g(a)=y|C_g(b)=x) =  p_g(a)^{x y} (1-p_g(a))^{x (1-y)} q_g(a)^{(1-x) y}(1-q_g(a))^{(1-x)(1-y)}, 
\end{equation*}
where $x, y \in \{0, 1\}$,
and the likelihood of~$c_g(a)$ given~$c_g(b)$
$$ L(c_g(a)|c_g(b)) = \P(C_g(a)=c_g(a)|C_g(b)=c_g(b)). $$

Assuming the independence of $L(c_g(a)|c_g(b))$ for different~$g$,
$$ L(a|b) = \prod_{g \in Genes} L(c_g(a)|c_g(b)). $$

Assuming the independence of $L(a|b)$ for different arcs~$(a,b)$,
the likelihood of the pan-genome phylogenetic tree 
$$ L = \prod_{(a,b) \in A} L(a|b). $$

For $(a,b) \in Arcs$ let
$$ len(a,b) \eqdef \sum_{g \in Genes} w (g,a,b,c_g(a),c_g(b)), $$
where 
$$ w (g,a,b,x,y) \eqdef - \log \P(C_g(a)=x|C_g(b)=y) $$
is the {\em arc length for one gene}.

Then
$$ len \eqdef - \log L = \sum_{(a,b) \in A} len(a,b). $$

The MLE of a pan-genome phylogenetic tree $(Nodes,Arcs)$ of $Genomes$ consists in
the selection of $Nodes$ and $Arcs$ 
such that $len$ is minimized.

Finding a MLE of a pan-genome phylogenetic tree is equivalent to solving a directed Steiner tree problem.

Since the Steiner tree problem is NP-hard, local optimization is used.

\comm {Rare genes have larger contribution to the arc lengths,
so a solution to the Steiner tree problem will tend to group the species, where these genes are core, into small subtrees.}



\subsection {Estimation of the gene content of nodes given a topology of the tree}


\subsubsection{Local optimization of~$\hat c_g(s)$}

Assume $p_g(s) \approx 1$, $q_g(s) = 1 - p_g(s)$ for all~$s$ and~$g$.

For a species~$s$ let~$N(s)$ be the {\em neighborhood} of~$s$:
$$ N(s) \eqdef \{t: s \to t \Or t \to s\}, $$
then 
the MLE of~$c_g(s)$ given $N(s)$
\begin{equation*}
\begin{split}
\hat c_g(s|N(s)) &= \arg \max_{c_g(s)} L(c_g(s)|N(s)) \\
   &= \arg \min_{c_g(s)} \sum_{t \in N(s)} (c_g(s) (1-c_g(t)) + (1-c_g(s)) c_g(t)) \\
   &= \arg \min_{c_g(s)} c_g(s) \left(|N(s)| - 2 \sum_{t \in N(s)} c_g(t) \right) \\
   &= 
  \begin{cases}
    \{0\},   &\text{ if } \sum_{t \in N(s)} c_g(t) < |N(s)|/2,\\
    \{1\},   &\text{ if } \sum_{t \in N(s)} c_g(t) > |N(s)|/2,\\
    \{0,1\}, &\text{ if } \sum_{t \in N(s)} c_g(t) = |N(s)|/2.
  \end{cases}   
\end{split}
\end{equation*}



\subsubsection{Global optimization of~$\hat c_g(s)$ and $len$}

Let $x, y \in \{0,1\}$.

Let $len_g(s)$ be the minimum tree length for gene~$g$ in the subgraph which consists of the subtree rooted at species~$s$ and the arc $(s,parent(s))$.
\begin{equation*}
  \begin{split}
    len_g(s,x)   &\eqdef len_g(s|c_g(parent(s)) = x). \\
    len_g(s,y,x) &\eqdef len_g(s|c_g(s) = y, c_g(parent(s)) = x). 
  \end{split}
\end{equation*}

\begin{equation*}
\hat c_g(s,x) \eqdef 
  \begin{cases}
    \{c_g(s)\}, &\text{ if $s$ is a genome},\\
    \arg \min_{y \in \{0,1\}} len_g(s,y,x), &\text{ else}.
  \end{cases}   
\end{equation*}
$$ \hat c_g(s,x) \in \{\{0\}, \{1\}, \{0,1\} \}. $$
$$ \forall s: \neg [\hat c_g(s,0) = \{0,1\} \And \hat c_g(s,1) = \{0,1\}]. $$  
\comm { prove }

\comm {
If~$s$ is an interior node then 
\begin{equation*}
\hat c_g(s,x) \eqdef 
  \begin{cases}
    y_1, &\text{ if } \hat C_g(s,x) = \{y_1\},\\
    y_2, &\text{ if } \hat C_g(s,x) = \{0,1\} \And \hat C_g(s, \bar x) = \{y_2\}.
  \end{cases}   
\end{equation*}
$$ \hat c_g(s,x) \in \{0, 1\}. $$
}

$$ len_g(s,x) = len_g(s, \iota \hat c_g(s,x), x). $$
$$ len_g(s,y,x) = w(g,s,parent(s),y,x) + \sum_{t \to s} len_g(t,y). $$
The values $len_g(s,x)$ and $\hat c_g(s,x)$ are stored at each node~$s$ for $x \in \{0,1\}$ and computed from leaves to the root.
After that $\hat c_g(s)$ are computed from the root to leaves:
\begin{equation*}
 \hat c_g(s) \eqdef 
  \begin{cases}
    c_g(s), &\text{ if $s$ is a genome},\\
    \iota \arg \min_{x \in \{0,1\}} len_g(s,x), &\text{ if } s = root,\\
    \iota \hat c_g(s, \hat c_g(parent(s))), &\text{ else},
  \end{cases}   
\end{equation*}
which may be ambiguous.
\comm {There are actually non-ambigous rules}

$$ len = \sum_g \min_{x \in \{0,1\}} len_g(root,x). $$

This is a Sankoff algorithm.  
\comm {ref}

If the root of the tree is changed to a different node of the tree, the resulting $len$ and $\hat c_g(s)$ will not change.
In this aspect it is an optimization on an unrooted tree.

The values $len_g(s,x)$ and $\hat c_g(s,x)$ can be quickly recomputed if the parent of node~$b$ is changed from~$a_1$ to~$a_2$, 
because it is enough to recompute these values for~$a_1$, $a_2$ and all their ancestors.

$$ \forall g,s,x : len_g (s,x) \le len_g(s,\bar x) + \max_{y \in \{0,1\}} (w(g,s,y,x) - w(g,s,y,\bar x)). $$
See~\cite{Mirkin03}.

\proof{
Use $\hat c_g$ corresponding to $len_g(s,\bar x)$ and change $\bar c_g(parent(s))$ from~$\bar x$ to~$x$.
That will increase $len_g(s,\bar x)$ by at most $\max_{y \in \{0,1\}} (w(g,s,y,x) - w(g,s,y,\bar x))$.
}

\comm
{
It is possible to compute $\hat c_g(s)$ replacing $len_g(s,x)$ and $\hat C_g(s,x)$, where $x\in \{0,1\}$, by $d_g(s) \in \{0,1,\nul\}$ at each node~$s$ as follows.

Assuming $\{0\} < \{0,1\} < \{1\}$, 
$$ \hat C_g(s,0) \le \hat C_g(s,1). $$  
\comm {prove}
$$ \hat c_g(s,0) \le \hat c_g(s,1). $$
\begin{equation*}
d_g(s) \eqdef 
  \begin{cases}
    \hat c_g(s,0), &\text{ if } \hat c_g(s,0) = \hat c_g(s,1),\\
    \nul, &\text{ else.}
  \end{cases}   
\end{equation*}
$$ d_g(s) \in \{0, 1, \nul \}. $$
$$ d_g(s) = \nul \equivalent \hat C_g(s,0) = \{0\} \And \hat C_g(s,1) = \{1\}. $$
$$ d_g(s) = \nul \implies len_g(s,x) = \sum_{t \to s} len_g(t,x). $$
$$ d_g(s) = \nul \implies len_g(s,0) = len_g(s,1), $$
which is proved by induction over~$s$ from leaves to the root.  
\comm {prove}
$$ len'_g(s,y,x) \eqdef |y - x| + \sum_{t \to s, d_g(t) \not= \nul} |d_g(t) - y|. $$
Then
$$ \hat C_g(s,x) = \arg \min_{y \in \{0,1\}} len'_g(s,y,x). $$

The values $d_g(s)$ are computed iterating over nodes~$s$ from leaves to the root.

After that
\begin{equation*}
\hat c_g(s,x) =   \begin{cases}
    x, &\text{ if } d_g(s) = \nul,\\
    d_g(s), &\text{ else.}
  \end{cases}   
\end{equation*}
}


\subsection {Time}

Let~$X \in \{0,1\}$ be a random Boolean attribute, which changes over time.
Let $\P_t(x_1|x_0)$ be the probability that $X = x_1$ after time~$t \ge 0$ given $X = x_0$ at $t=0$.

Let $\Delta t \approx 0$ and
$$ \pi_x \eqdef \P_{\Delta t}(\bar x|x), $$
then for a Markov chain
$$ \P_t(\bar x|x) = \frac {\pi_x} {\pi_x + \pi_{\bar x}} \left(1 - (1 - (\pi_x + \pi_{\bar x}))^{t / \Delta t} \right). $$

Assume the time needed to change from state~$x$ to~$\bar x$ is distributed as $Exponential(\lambda_x)$, where $\lambda_x > 0$,
the mean being $1 / \lambda_x$,
then
$$ \pi_x = 1 - e^{- \lambda_x \Delta t}. $$

$$ \lambda \eqdef \lambda_0 + \lambda_1. $$

Since
$$ \lim_{\alpha \to 0} \frac {e^\alpha} {1 + \alpha} = 1, $$

$$ \pi_x = \lambda_x \Delta t, $$
$$ \frac {\pi_x} {\pi_x + \pi_{\bar x}} = \frac {\lambda_x} \lambda. $$

$$ (1 - (\pi_x + \pi_{\bar x}))^{t / \Delta t} = e^{-(\pi_x + \pi_{\bar x}) \ t / \Delta t}= e^{-\lambda t}.$$

Thus,
$$ \P_t(\bar x|x) = \frac {\lambda_x} \lambda \left(1 -  e^{-\lambda t} \right). $$

$$ \lim_{t \to 0} \frac {\P_t(\bar x|x)} {1 - e^{- \lambda_x t}} = \lim_{t \to 0} \frac {\P_t(\bar x|x)} {\lambda_x t} = 1. $$

$$ \lim_{t \to \infty} \P_t(\bar x|x) = \frac {\lambda_x} \lambda. $$

$$ \P_{t_1 + t_2}(\bar x|x) = \P_{t_1}(\bar x|x) \P_{t_2}(\bar x|\bar x) + \P_{t_1}(x|x) \P_{t_2}(\bar x|x) 
  = \frac {\lambda_x} \lambda \left(1 -  e^{-\lambda (t_1 + t_2)} \right). 
$$

$$ \forall t: \frac {\P_t(\bar x|x)} {\P_t(x|\bar x)} = \frac {\lambda_x} {\lambda_{\bar x}}. $$

Since time units are arbitrary we can set $\lambda = 1$.

$$ \P(1) = \P_\infty(1|0) \times \P(0) + (1 - \P_\infty(0|1)) \times \P(1) = \lambda_0 \P(0) + (1 - \lambda_1) \P(1) = \lambda_0. $$

Let $t(s)$ be the time between nodes~$s$ and $parent(s)$,
$x$~be $c_g(s)$, 
and $\lambda_x(g)$ be~$\lambda_x$ for gene~$g$,
then
\begin{equation*} 
\begin{split}
\P (g \in Core(b) | g \in Core(a)) &= 1 - \P_t(0|1) = 1 - \lambda_1(g) \left(1-e^{-t(b)} \right), \\
\P (g \in Core(b) | g \not \in Core(a)) &= \P_t(1|0)     = \lambda_0(g) \left(1-e^{-t(b)} \right). 
\end{split}
\end{equation*}

It will be assumed in the sequel that $\lambda_x(g)$ do not depend on~$g$.


\subsection {Maximum parsimony method}

If $t(s) = t \approx 0$ at each node~$s$,
then $ \P_t(\bar x|x) = \lambda_x t$, $ \P_t(x|x) = 1$
and the equations of Section~\ref{MLECriterion} get simplified as
$$ \P(C_g(b)=y|C_g(a)=x) = (\lambda_1 t)^{x (1-y)} (\lambda_0 t)^{(1-x) y} $$
and
$$ w(g,s,y,x) = (- \log (\lambda_1 t)) x (1-y) + (- \log (\lambda_0 t)) (1-x) y, $$
which does not depend on~$g$ and~$s $.

Let
$$ n_{yx} \eqdef \sum_{g, \ b \to a}  \neg^{\bar y} c_g(b) \times \neg^{\bar x} c_g(a), $$
then 
\begin{equation*} 
\begin{split}
 len &= \sum_{y,x} w(g,s,y,x) \ n_{yx} \\
     &= (- \log (\lambda_1 t)) \ n_{01} + (- \log (\lambda_0 t)) \ n_{10}  \\
     &= (-\log \lambda_1) \ n_{01} + (- \log \lambda_0) \ n_{10} + (- \log t) (n_{01} + n_{10}) \\
     &= const \times (n_{01} + n_{10}).
\end{split}
\end{equation*}

The value $n_{01} + n_{10}$ will be referred to as {\em amount of evolution},
which can be computed as $len$, where $w(g,s,y,x)$ is replaced by $|x - y|$.

Minimizing the amount of evolution is the {\em maximum parsimony method}.

If the number of genomes tends to $\infty$ then $t(s) \to 0$ for all~$s$ and MLE becomes equivalent to the maximum parsimony method.

$$ \forall g,s,x : |len_g (s,x) - len_g(s,\bar x)| \le 1. $$

\comm {  
Let a {\em singleton} be a gene which belongs to only one genome.
Removing singletons does not change the solution to the Steiner tree problem.
Removing the intersection core also does not change the solution to the Steiner tree problem.
}


\subsection {MLE of parameters}
Let $n_{xy}$ be the number of Boolean attributes which change their state from~$y$ to~$x$ over time~$t > 0$, where $x,y \in \{0,1\}$.
Then the likelihood
$$ L = \prod_{x,y} \P_t^{n_{xy}}(x|y). $$
$$ l = - \log L = - \sum_c \left(   n_{\bar c c} \log      \P_t(\bar c|c) 
                                  + n_{     c c} \log (1 - \P_t(\bar c|c)) \right). $$
$$ \frac {d l} {d t} = 0. $$
\begin{equation} \label{t_mle}
\sum_c n_{\bar c c} \frac {\P'_t(\bar c|c)} {\P_t(\bar c|c)} = \sum_c n_{c c} \frac {\P'_t(\bar c|c)} {1 - \P_t(\bar c|c)}. 
\end{equation}

The MLE of~$t$ is the solution to
$$ \frac 1 {1 - e^{-t}} \sum_c n_{\bar c c} = \sum_c \frac {n_{cc}}  {\frac 1 {\lambda_c} - (1 - e^{-t})}. $$
The LHS decreases in~$t$, the RHS increases in~$t$, therefore, $\hat t$ can be found by binary search.
If the time is reversed, i.~e., the child and parent species are swapped, $\hat t$ will be the same.

The MLE of~$\lambda_0$ is the solution to
$$   n_{10} \frac 1 {\lambda_0} + n_{11} \frac {1 - e^{-t}} {1 - \lambda_1 (1 - e^{-t})} 
   = n_{01} \frac 1 {\lambda_1} + n_{00} \frac {1 - e^{-t}} {1 - \lambda_0 (1 - e^{-t})}.
$$
The LHS decreases in~$\lambda_0$, the RHS increases in~$\lambda_0$, therefore, $\lambda_0$ can be found by binary search.
If $t=0$ then $\hat \lambda_0 = n_{10} / (n_{10} + n_{01})$.
If $t=\infty$ then $\hat \lambda_0 = (n_{10} + n_{11}) / (n_{10} + n_{11} + n_{01} + n_{00})$.



\subsection {Annotation probability}

\comm {To be corrected}
Let $\pi_a(i)$ be the probability that a gene in genome~$i$ is annotated if it exists in $Core(i)$. 
Usually $\pi_a(i) \approx 1$.
Let $\pi_a(i) \eqdef 1$, if $i$ is not a genome.

Then equations~(\ref{pq}) are changed to:
\begin{equation*}
\begin{split}
p_g(b) &\eqdef \pi_a(b) \ \P (g \in Core(b) | g \in Core(a)), \\
q_g(b) &\eqdef \pi_a(b) \ \P (g \in Core(b) | g \not \in Core(a)). 
\end{split}
\end{equation*}

\comm {
In the maximum parsimony method, 
assuming $\pi_a(s) = \pi_a$ for all~$s$, 
$len$ is changed to
$$ len = (-\log \pi_a) \ n_{11} + (- \log (\lambda_1 t)) \ n_{01} + (- \log (\pi_a \lambda_0 t)) \ n_{10},  $$
which remains proportional to the amount of evolution.
}

Let
$$ Q_t(\bar c|c) \eqdef \lambda_c \left(1 -  e^{-t} \right), $$
then
$$ \P_t(x|y) = \neg^{\bar x} (\neg^y Q_t(\bar y|y) \pi_a), $$
\comm { $$ \frac d {dt} Q_t(\bar y|y) = \lambda_y e^{-t}, $$}
$$ \frac d {dt} \P_t(x|y) = (-1)^{\bar x + y} \pi_a \lambda_y e^{-t}, $$
and, according to~(\ref{t_mle}), the MLE of~$t$ is the solution to
$$   \sum_c n_{\bar c c} \frac {\lambda_c} {\neg^c (\pi_a \neg^c (\lambda_c (1 - e^{-t}))) }
   = \sum_c n_{c c} \frac {\lambda_c} {\neg^{c+1} (\pi_a \neg^c (\lambda_c (1 - e^{-t}))) }. $$


\subsection {Root}
Since MLE results equal the true parameters asymptotically, 
the tree root cannot be estimated well by MLE, if its neighborhood in the tree is small.

If a gene is present in a node, but absent in a cut of the subtree rooted at this node, then this gene cannot be estimated in that node by the Sankoff algorithm.
\comm { because unchange weight is less than change weight.}
The farther from leaves a node is, 
the higher the probability that this event occurs
and the more the number of the genes assigned to this node is biased towards a smaller number.
This bias is biggest at the root.
For an arc which is near the root, the time is biased towards a smaller value.

A node created in the middle of an arc which has the smallest number of genes is a candidate for the root.


\section{Maximum parsimony tree}

Consider a maximum parsimony tree.

If $c_g(s)$ then there is a genome~$i$, s.t.~$i$ is a descendant of~$s$ and $\forall j \in path(i,s) : c_g(j)$.

\comm{
$$ (\forall i \in \text{Genomes} : \neg (c_g(i) \And c_h(i))) \implies (\forall i :  \neg (c_g(i) \And c_h(i))). $$
\proof
{Let $c_g(i) \And c_h(i)$ and~$i$ have the maximum depth.
}
}



\section{Phenotypes}
Let~$p$ be a {\em phenotype} which is a Boolean feature of genomes.
For all genomes a phenotype must be either identified or flagged as unknown.
Let~$g(p)$ be an abstract gene which is in a genome whenever the genome has phenotype~$p$.

Inter-species phenotypes:
\begin{itemize}
	\item Genetic code
	\item Number of chromosomes
	\item Topology of chromosomes: circular, linear
	\item Chromosome size
	\item GC \% (depends on temperature?)
	\item Number and size of plasmids
	\item Variant and operon location of ribosomal proteins
	\item Number of genes
	\item Symbiont
	\item Anatomy:
	\begin{itemize}
		\item Cell wall
		\item Shape
		\item Motility
	\end{itemize}
	\item \{Chemo$|$photo\}\{auto$|$hetero\}trophness.
    \item Environment:
	\begin{itemize}
		\item Oxygen requirement
		\item Temperature 
		\item pH 
		\item Halophilicity
		\item Habitat
	\end{itemize}
\end{itemize}


Intra-species phenotypes:
\begin{itemize}
  \item Plasmids
  \item Pathogenicity
\end{itemize}


\Questions
\begin{itemize}
\item Missing characterizing genes due to location in a small island and incompleteness of a genome. 
      Genome will tend to move to the root of its subtree (example: 2 genomes of Montevideo)?
\item Divergence of core in outbreaks, e.g., in Salmonella. Odd genes?
\item Pseudogenization. Shows the direction of arcs. Mutation $g \to a$ preserves stop codons.
\item Possibility of convergence of species: $a \ne b$, but $Core(a) = Core(b)$.
\item Order (maybe circularly) the ORFs of $Core(s)$.
\item Dependence of $\lambda_c$ on a species.
\item Dependence of $\lambda_c$ on a gene.
\item Dependence of $len_{g_1}(s,x)$ on $len_{g_2}(s,x)$.
\item Total time on each path from a node to a leaf should be the same.
\end{itemize}


\section {Comparison of the topologies of phylogenetic trees}
Let~$T$ be a phylogenetic tree over a set of genomes.
Let~$P$ be a set of phenotypes.

For $p \in P$, let $n(p)$ be the number of nodes~$s$ in tree~$T$ where $C_{g(p)}(s) \ne C_{g(p)}(parent(s))$.
$$ n(p) \ge len_{g(p)}(root), $$
where $len$ is computed for this tree by the maximum parsimony method.

Then define the {\em inconsistency} of tree~$T$ as 
$$ inc(T) \eqdef \sum_{p \in P} n(p). $$

Tree~$T_1$ is better than tree~$T_2$ if $inc(T_1) < inc(T_2)$.

If $C_{g(p)}$ is not available for $T_2$, $n(p)$ can be replaced by its lower bound in $inc(T_2)$.


\section {Phenotype-gene association}

Let $s \in Species$, 
$G(s)$ be the genes of~$s$,
$P(s)$ be the phenotypes of~$s$,
$$ X^+(s) \eqdef X(s) \setminus X(parent(s)), $$
$$ X^-(s) \eqdef X(parent(s)) \setminus X(s), $$
where $X$ is $G$ or $P$.

Suppose the gene function of a phenotype~$p$ is a conjunction of literals of genes.
Let $G^+(p)$ and $G^-(p)$ be the sets of genes in the positive and negative literals respectfully.
\comm { $\neg G^+(p) \bot G^-(p)$.}
Then
$$ \forall s: \ p \in P^+(s) \implies G^+(s) \bot G^+(p) \Or G^-(s) \bot G^-(p), $$
$$ \forall s: \ p \in P^-(s) \implies G^+(s) \bot G^-(p) \Or G^-(s) \bot G^+(p). $$

The smaller $|G^{+/-}(s)|$, the smaller $|G^{+/-}(p)|$ .

How to reduce $|G^{+/-}(s)|$:
\begin{itemize}
  \item Sample more genomes in the neighborhood of~$s$.
  \item Remove the genes, which are not in the pathway of~$p$, from $G^{+/-}(s)$.
\end{itemize}

\comm {}
An evolution tree tends to form subtrees with different phenotypes.
This facilitates the gene-phenotype association by an evolution tree.


\section {Gene-node association}

A gene in phylogenetically connected nodes $\equivalent$ very similar sequence, same genome context and same locus.



\section{Mobilome}
{\em Mobilome} are mobile genetic elements in a bacterial genome.

{\em Bacterial genome} is chromosomes and plasmids.

{\em Phage} is a virus that replicates within bacteria.

{\em Insertion sequence} element, or an {\em IS} \ element is a minimal transposon.

Types of mobilome:
\begin{itemize}
  \item Plasmid. Conjugation by {\em tra} genes.
  \item Prophage: phage genome in bacterial genome. 
        \\
        Genomic islands:
        \begin{itemize}
          \item Higher G+C content.
          \item Atypical nucleotide frequencies.
        \end{itemize}
        Insertion elements.
  \item Transposon
\end{itemize}

\begin{tabular}{|c|c|}
\hline Mobilome & HGT \\
\hline 
\hline Plasmid & Conjugation \\
\hline Bacteriophage & Transduction \\
\hline Insertion elements & Transformation \\
\hline
\end{tabular}
\\


Prophage and transposon genes are {\em odd}.


\Questions
\begin{itemize}
  \item Specific pathways for HGT (?)
  \item \# Odd genes in a species:
    \begin{itemize}
  	  \item Distribution of \# odd genes in a genome of a species.
    \end{itemize}
\end{itemize}



\section{TODO}

\begin{itemize}
  \item Difference between core and accessory genes:
    \begin{itemize}
   	  \item GC \%
	  \item Distribution of E. coli proteins per COG function:
	  \begin{itemize}
    	  \item Proteobacteria and prokaryotes are similar.
    	  \item Core is similar to proteobacteria, especially in the classes ``information" and ``cellular".
    	  \item Core is more similar to proteobacteria than to prokaryotes.
    	  \item Accessory is not similar to proteobacteria or prokaryotes.
    	  \item Functions ``Transposase", ``Integrase", ``Recombinase", ``Chromatin structure and dynamics", ``Type II secretory pathway" almost do not occur in core relative to accessory.
    	  \item Accessory is enriched relative to core in functions ``Replication, recombination and repair", ``Cell motility", ``Secretory pathway", ``Intracellular trafficking, secretion, and vesicular transport", ``Extracellular structures".
	  \end{itemize}
    \end{itemize}
 	\item The distribution of odd genes vs. the distribution of all genes in prokaryotes.
 	\item The distribution of accessory genes should be the same as the distribution of odd genes.
 	\item Independence of $p_{accessory}(g)$ from species. Dependence of $p_{accessory}(g)$ on gene length, etc.
	\item Method to find the species for a genome by comparing with core genes of different species (decision tree, similar to taxonomy).
    \item Zipf distribution of subspecies size for a species.
    \item Node-parent stability measured by difference of $len$ due to changing the parent (cf. ``bootstrap"). 
          \par Aggregation of stabilities --- tree quality.
\end{itemize}


\begin{thebibliography}{9}

\bibitem{Mirkin03}
Boris G.~Mirkin, Trevor I.~Fenner, Michael Y.~Galperin, and Eugene V.~Koonin, 
\emph{Algorithms for computing parsimonious evolutionary scenarios for genome evolution, the last universal common ancestor and dominance of horizontal gene transfer in the evolution of prokaryotes,}
BMC Evol Biol. 2003; 3: 2.
Published online Jan 6, 2003. doi:  10.1186/1471-2148-3-2
PMCID: PMC149225

\end{thebibliography}


\end{document}


